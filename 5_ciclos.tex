% Copyright (C) Rosita Wachenchauzer <rositaw@gmail.com>
% Copyright (C) Margarita Manterola <margamanterola@gmail.com>

% Esta obra está licenciada de forma dual, bajo las licencias Creative
% Commons:
%  * Atribución-Compartir Obras Derivadas Igual 2.5 Argentina
%    http://creativecommons.org/licenses/by-sa/2.5/ar/
%  * Atribución-Compartir Obras Derivadas Igual 3.0 Unported
%    http://creativecommons.org/licenses/by-sa/3.0/deed.es_AR.
%
% A su criterio, puede utilizar una u otra licencia, o las dos.
% Para ver una copia de las licencias, puede visitar los sitios
% mencionados, o enviar una carta a Creative Commons,
% 171 Second Street, Suite 300, San Francisco, California, 94105, USA.

\chapter{Más sobre ciclos}

El último problema analizado en la unidad anterior decía:

\begin{quote}
Leer un número. Si el número es positivo escribir un mensaje
``Numero positivo'', si el número es igual a 0 un mensaje ``Igual a
0'', y si el número es negativo escribir un mensaje ``Numero
negativo''.
\end{quote}

Se nos plantea a continuación un nuevo problema, similar al anterior:

\problemac{El usuario debe poder ingresar muchos números y
cada vez que se ingresa uno debemos informar si es positivo, cero o
negativo.} \\

Utilizando los ciclos definidos vistos en las primeras unidades, es posible
preguntarle al usuario cada vez, al inicio del programa, cuántos números va a
ingresar para consultar. La solución propuesta resulta:

\begin{codigo-python}
def muchos_pcn():
    i = input("Cuantos numeros quiere procesar: ")
    for j in range(0,i):
        x = input("Ingrese un numero: ")
        if x > 0:
            print "Numero positivo"
        elif x == 0:
            print "Igual a 0"
        else:
            print "Numero negativo"
\end{codigo-python}

Su ejecución es exitosa:

\begin{codigo-python-sn}
>>> muchos_pcn()
Cuantos numeros quiere procesar: 3
Ingrese un numero: 25
Numero positivo
Ingrese un numero: 0
Igual a 0
Ingrese un numero: -5
Numero negativo
>>>
\end{codigo-python-sn}

Sin embargo al usuario considera que este programa no es muy intuitivo, porque
lo obliga a contar de antemano cuántos números va a querer procesar, sin
equivocarse, en lugar de ingresar uno a uno los números hasta procesarlos a
todos.

\section{Ciclos indefinidos}

Para poder resolver este problema sin averiguar primero la cantidad de números
a procesar, debemos introducir una instrucción que nos permita construir ciclos
que no requieran que se informe de antemano la cantidad de veces que se
repetirá el cálculo del cuerpo.  Se trata de {\it ciclos indefinidos} en los
cuales se repite el cálculo del cuerpo mientras una cierta condición es
verdadera.

Un ciclo indefinido es de la forma

\begin{codigo-python-sn}
while <condición>:
    <hacer algo>
\end{codigo-python-sn}

Donde \lstinline+while+ es una palabra reservada, la condición es una expresión
booleana, igual que en las instrucciones \lstinline!if!.  Y el cuerpo es, como
siempre, una o más instrucciones de Python.

El sentido de esta instrucción es el siguiente:

\begin{enumerate}
\item Evaluar la condición.
\item Si la condición es falsa, salir del ciclo.
\item Si la condición es verdadera, ejecutar el cuerpo.
\item Volver a 1.
\end{enumerate}

\section{Ciclo interactivo}

¿Cuál es la condición y cuál es el cuerpo del ciclo en nuestro problema?
Claramente, el cuerpo del ciclo es el ingreso de datos y la verificación de si
es positivo, negativo o cero.  En cuanto a la condición, es que haya más datos
para seguir calculando.

Definimos una variable \lstinline!hayMasDatos!, que valdrá ``Si'' mientras
haya datos.

Se le debe preguntar al usuario, después de cada cálculo, si hay o no más datos.
Cuando el usuario deje de responder ``Si'', dejaremos de ejecutar el cuerpo del
ciclo.

Una primera aproximación al código necesario para resolver este problema podría
ser:

\begin{codigo-python}
def pcn_loop():
    while hayMasDatos == "Si":
        x = input("Ingrese un numero: ")
        if x > 0:
            print "Numero positivo"
        elif x == 0:
            print "Igual a 0"
        else:
            print "Numero negativo"

        hayMasDatos = raw_input("¿Quiere seguir? <Si-No>: ")
\end{codigo-python}

Veamos qué pasa si ejecutamos la función tal como fue presentada:

\begin{codigo-python-sn}
>>> pcn_loop()

Traceback (most recent call last):
  File "<pyshell#25>", line 1, in <module>
    pcn_loop()
  File "<pyshell#24>", line 2, in pcn_loop
    while hayMasDatos == "Si":
UnboundLocalError: local variable 'hayMasDatos' referenced before assignment
>>>
\end{codigo-python-sn}

El problema que se presentó en este caso, es que \lstinline!hayMasDatos! no
tiene un valor asignado en el momento de evaluar la condición del ciclo por
primera vez.

\begin{observacion}
Es importante prestar atención a cuáles son las variables que hay que
inicializar antes de ejecutar un ciclo: al menos tiene que tener algún valor la
expresión booleana que lo controla.
\end{observacion}

Una posibilidad es preguntarle al usario, antes de evaluar la condición, si
tiene datos; otra posibilidad es suponer que si llamó a este programa es porque
tenía algún dato para calcular, y darle el valor inicial ``Si'' a
\lstinline!hayMasDatos!.

Acá encararemos la segunda posibilidad:

\begin{codigo-python}
def pcn_loop():
    hayMasDatos = "Si"
    while hayMasDatos == "Si":
        x = input("Ingrese un numero: ")
        if x > 0:
            print "Numero positivo"
        elif x == 0:
            print "Igual a 0"
        else:
            print "Numero negativo"

        hayMasDatos = raw_input("Quiere seguir? <Si-No>: ")
\end{codigo-python}

El esquema del ciclo interactivo es el siguiente:
\begin{itemize}
\item \lstinline!hayMasDatos! hace referencia a ``Si''.
\item Mientras \lstinline!hayMasDatos! haga referencia a ``Si'':

\begin{itemize}
\item Pedir datos.
\item Realizar cálculos.
\item Preguntar al usuario si hay más datos (``Si'' cuando los hay).
\lstinline!hayMasDatos! hace referencia al valor ingresado.
\end{itemize}

\end{itemize}

Ésta es una ejecución:

\begin{codigo-python-sn}
>>> pcn_loop()
Ingrese un numero: 25
Numero positivo
Quiere seguir? <Si-No>: "Si"
Ingrese un numero: 0
Igual a 0
Quiere seguir? <Si-No>: "Si"
Ingrese un numero: -5
Numero negativo
Quiere seguir? <Si-No>: "No"
>>>
\end{codigo-python-sn}

\section{Ciclo con centinela}
\label{centinela}

Un problema que tiene nuestra primera solución es que resulta poco amigable
preguntarle al usuario después de cada cálculo si desea continuar. Se puede
usar el método del {\it centinela}: un valor distinguido que, si se lee, le
indica al programa que el usuario desea salir del ciclo. En este caso,
podemos suponer que si ingresa el caracter \lstinline!*! es una indicación
de que desea terminar.

El esquema del ciclo con centinela es el siguiente:

\begin{itemize}
\item Pedir datos.
\item Mientras el dato pedido no coincida con el centinela:
\begin{itemize}
\item Realizar cálculos.
\item Pedir datos.
\end{itemize}
\end{itemize}

En nuestro caso, pedir datos corresponde a lo siguiente:

\begin{itemize}
\item Pedir número.
\end{itemize}

El programa resultante es el siguiente:

\begin{codigo-python}
def pcn_loop2():
    x=input("Ingrese un numero ('*' para terminar): ")

    while x <>"*":
        if x > 0:
            print "Numero positivo"
        elif x == 0:
            print "Igual a 0"
        else:
            print "Numero negativo"

        x=input("Ingrese un numero ('*' para terminar): ")
\end{codigo-python}

Y ahora lo ejecutamos:

\begin{codigo-python-sn}
>>> pcn_loop2()
Ingrese un numero ('*' para terminar): 25
Numero positivo
Ingrese un numero ('*' para terminar): 0
Igual a 0
Ingrese un numero ('*' para terminar): -5
Numero negativo
Ingrese un numero ('*' para terminar): '*'
>>>
\end{codigo-python-sn}

\section{Cómo romper un ciclo}

El ciclo con centinela es muy claro pero tiene un problema: hay dos lugares
(la primera línea del cuerpo y la última línea del ciclo) donde se ingresa
el mismo dato. Si en la etapa de mantenimiento tuviéramos que realizar un
cambio en el ingreso del dato (cambio de mensaje, por ejemplo) deberíamos
estar atentos y hacer dos correcciones iguales.

Sería preferible poder leer el dato \lstinline!x! en un único punto del
programa.  A continuación, tratamos de diseñar una solución con esa
restricción.

Es claro que en ese caso la lectura tiene que estar dentro del ciclo para
poder leer más de un número, pero entonces la condición del ciclo no puede
depender del valor leído, ni tampoco de valores calculados dentro del
ciclo.

Pero un ciclo que no puede depender de valores leídos o calculados dentro
de él será de la forma:

\begin{itemize}
\item Repetir indefinidamente:
\begin{itemize}
\item Hacer algo.
\end{itemize}
\end{itemize}

Y esto se traduce a Python como:

\begin{codigo-python-sn}
while True:
    <hacer algo>
\end{codigo-python-sn}

Un ciclo cuya condición es \lstinline!True! parece ser un ciclo infinito (o
sea que nunca va a terminar). ¡Pero eso es gravísimo! ¡Nuestros programas
tienen que terminar!

Afortunadamente hay una instrucción de Python, \lstinline!break!, que nos
permite salir de adentro de un ciclo (tanto sea \lstinline!for! como
\lstinline!while!) en medio de su ejecución.

En esta construcción

\begin{codigo-python-sn}
while <condicion>:
    <hacer algo_1>
    if <condif>:
        break
    <hacer algo_2>
\end{codigo-python-sn}

el sentido del \lstinline!break! es el siguiente:

\begin{enumerate}
\item Se evalúa {\it $<$condición$>$} y si es falsa se sale del ciclo.
\item Se ejecuta {\it $<$hacer algo$_1>$}.
\item Se evalúa {\it $<$condif$>$} y si es verdadera se sale del ciclo (con \lstinline!break!).
\item Se ejecuta {\it $<$hacer algo$_2>$}.
\item Se vuelve al paso 1.
\end{enumerate}

Diseñamos entonces:

\begin{itemize}
\item Repetir indefinidamente:
\begin{itemize}
\item Pedir dato.
\item Si el dato ingresado es el centinela, salir del ciclo.
\item Operar con el dato.
\end{itemize}
\end{itemize}

Codificamos en Python la solución al problema de los números usando ese
esquema:

\begin{codigo-python}
def pcn_loop3():
    while True:
        x = input("Ingrese un numero ('*' para terminar): ")
        if x == '*':
            break
        elif x > 0:
            print "Numero positivo"
        elif x == 0:
            print "Igual a 0"
        else:
            print "Numero negativo"
\end{codigo-python}

Y la probamos:

\begin{codigo-python-sn}
>>> pcn_loop3()
Ingrese un numero ('*' para terminar): 25
Numero positivo
Ingrese un numero ('*' para terminar): 0
Igual a 0
Ingrese un numero ('*' para terminar): -5
Numero negativo
Ingrese un numero ('*' para terminar): '*'
>>>
\end{codigo-python-sn}

\begin{sabias_que}
Desde hace mucho tiempo los ciclos infinitos vienen trayéndoles dolores de
cabeza a los programadores.  Cuando un programa deja de responder y se
queda utilizando todo el procesador de la computadora, suele deberse a que
el programa entró en un ciclo del que no puede salir.

Estos ciclos pueden aparecer por una gran variedad de causas.  A
continuación algunos ejemplos de ciclos de los que no se puede salir,
siempre o para ciertos parámetros.  Queda como ejercicio encontrar el error
en cada uno.

\begin{codigo-python-sn}
def menor_factor_primo(x):
    """ Devuelve el menor factor primo del número x. """
    n = 2
    while n <= x:
        if x % n == 0:
            return n
\end{codigo-python-sn}

\begin{codigo-python-sn}
def buscar_impar(x):
    """ Divide el número recibido por 2 hasta que sea impar. """
    while x % 2 == 0:
        x = x / 2
    return x
\end{codigo-python-sn}
\end{sabias_que}

\section{Ejercicios}

\ejercicioc{Nuevamente, se desea facturar el uso de un telefono.  Para ello
se informa la tarifa por segundo y la duracion de cada comunicacion
expresada en horas, minutos y segundos.  Como resultado se informa la
duracion en segundos de cada comunicacion y su costo. Resolver este
problema usando

\begin{enumerate}
\item Ciclo definido.
\item Ciclo interactivo.
\item Ciclo con centinela.
\item Ciclo ``infinito'' que se rompe.
\end{enumerate}
}

\ejercicioc{Mantenimiento del tarifador: al final del día se debe informar
cuántas llamadas hubo y el total facturado. Hacerlo con todos los esquemas
anteriores.}

\ejercicioc{Nos piden que escribamos una función que le pida al usuario que
ingrese un número positivo. Si el usuario ingresa cualquier cosa que no sea
lo pedido se le debe informar de su error mediante un mensaje y volverle a
pedir el número.

Resolver este problema usando

\begin{enumerate}
\item Ciclo interactivo.
\item Ciclo con centinela.
\item Ciclo ``infinito'' que se rompe.
\end{enumerate}

¿Tendría sentido hacerlo con ciclo definido? Justificar.
}

\newpage
\section{Resumen}

\begin{itemize}

\item Además de los ciclos definidos, en los que se sabe cuáles son los
posibles valores que tomará una determinada variable, existen los ciclos
indefinidos, que se terminan cuando no se cumple una determinada condición.

\item La condición que termina el ciclo puede estar relacionada con una entrada
de usuario o depender del procesamiento de los datos.

\item Se utiliza el método del {\it centinela} cuando se quiere que un ciclo se
repita hasta que el usuario indique que no quiere continuar.

\item Además de la condición que hace que el ciclo se termine, es posible
interrumpir su ejecución con código específico dentro del ciclo.

\end{itemize}

\begin{referencia_python}

\begin{sintaxis}{\lstinline!while <condicion>:!}
Introduce un ciclo indefinido, que se termina cuando la condición sea falsa.
\begin{codigo-python-sn}
while <condición>:
    # acciones a ejecutar mientras condición sea verdadera
\end{codigo-python-sn}
\end{sintaxis}

\begin{sintaxis}{\lstinline!break!}
Interrumpe la ejecución del ciclo actual. Puede utilizarse tanto para ciclos
definidos como indefinidos.
\end{sintaxis}

\end{referencia_python}

\newpage
\section{Ejercicios}

\begin{ejercicio}
Escribir un programa que reciba una a una las notas del usuario,
preguntando a cada paso si desea ingresar más notas, e imprimiendo el
promedio correspondiente.
\end{ejercicio}

\begin{ejercicio}
Escribir una función que reciba un número entero $k$ e imprima su
descomposición en factores primos.
\end{ejercicio}

\begin{ejercicio}
{\bf Manejo de contraseñas}
\begin{partes}
    \item Escribir un programa que contenga una contraseña inventada, que le
pregunte al usuario la contraseña, y no le permita continuar hasta que la
haya ingresado correctamente.
    \item Modificar el programa anterior para que solamente permita una
cantidad fija de intentos.
    \item Modificar el programa anterior para que después de cada intento
agregue una pausa cada vez mayor, utilizando la función \verb!sleep! del
módulo \verb!time!.
    \item Modificar el programa anterior para que sea una función que devuelva
si el usuario ingresó o no la contraseña correctamente, mediante un valor
booleano (True o False).
\end{partes}
\end{ejercicio}


\begin{ejercicio}
Utilizando la función \verb!randrange! del módulo \verb!random!,
escribir un programa que obtenga un número aleatorio secreto, y luego
permita al usuario ingresar números y le indique sin son menores o mayores
que el número a adivinar, hasta que el usuario ingrese el número correcto.
\end{ejercicio}


\begin{ejercicio}
{\bf Algoritmo de Euclides}
\begin{partes}
    \item Implementar en python el algoritmo de Euclides para calcular el máximo
común divisor de dos números $n$ y $m$, dado por los siguientes pasos.
    \begin{enumerate}
        \item Teniendo $n$ y $m$, se obtiene $r$, el resto de la
división entera de $m/n$.
        \item Si $r$ es cero, $n$ es el mcd de los valores iniciales.
        \item Se reemplaza $m \leftarrow n$, $n \leftarrow r$, y se vuelve al
primer paso.
    \end{enumerate}
    \item Hacer un seguimiento del algoritmo implementado para los siguientes
pares de números: (15,9); (9,15); (10,8); (12,6).
\end{partes}
\end{ejercicio}

\begin{ejercicio}
{\bf Potencias de dos.}
\begin{partes}
    \item Escribir una función \verb!es_potencia_de_dos! que reciba como parámetro
un número natural, y devuelva \verb!True! si el número es una potencia de 2,
y \verb!False! en caso contrario.
    \item Escribir una función que, dados dos números naturales pasados como
parámetros, devuelva la suma de todas las potencias de 2 que hay en el
rango formado por esos números (0 si no hay ninguna potencia de 2 entre los
dos). Utilizar la función \verb!es_potencia_de_dos!, descripta en el
punto anterior.
\end{partes}
\end{ejercicio}


\begin{ejercicio}
{\bf Números perfectos y números amigos}
\begin{partes}
    \item Escribir una función que devuelva la suma de todos los divisores de
un número $n$, sin incluirlo.
    \item Usando la función anterior, escribir una función que imprima los
primeros $m$ números tales que la suma de sus divisores sea igual a sí
mismo (es decir los primeros $m$ números {\it perfectos}).
    \item Usando la primera función, escribir una función que imprima las
primeras $m$ parejas de números ($a$,$b$), tales que la suma de los
divisores de $a$ es igual a $b$ y la suma de los divisores de $b$ es igual
a $a$ (es decir las primeras $m$ parejas de números {\it amigos}).
    \item Proponer optimizaciones a las funciones anteriores para disminuir el
tiempo de ejecución.
\end{partes}
\end{ejercicio}

\begin{ejercicio}
Escribir un programa que le pida al usuario que ingrese una sucesión
de números naturales (primero uno, luego otro, y así hasta que el
usuario ingrese '-1' como condición de salida). Al final, el programa
debe imprimir cuántos números fueron ingresados, la suma total de los
valores y el promedio.
\end{ejercicio}

\begin{ejercicio}
Escribir una función que reciba dos números como parámetros, y
devuelva cuántos múltiplos del primero hay, que sean menores que el
segundo.
\begin{partes}
    \item Implementarla utilizando un ciclo \verb!for!, desde el primer número
hasta el segundo.
    \item Implementarla utilizando un ciclo \verb!while!, que multiplique el primer
número hasta que sea mayor que el segundo.
    \item Comparar ambas implementaciones: ¿Cuál es más clara? ¿Cuál realiza menos
operaciones?
\end{partes}
\end{ejercicio}

\begin{ejercicio}
Escribir una función que reciba un número natural e imprima todos
los números primos que hay hasta ese número.
\end{ejercicio}

\begin{ejercicio}
Escribir una función que reciba un dígito y un número natural, y
decida numéricamente si el dígito se encuentra en la notación decimal
del segundo.
\end{ejercicio}

\begin{ejercicio}
Escribir una función que dada la cantidad de ejercicios de un examen,
y el porcentaje necesario de ejercicios bien resueltos necesario para aprobar
dicho examen, revise un grupo de examenes. Para ello, en cada paso debe
preguntar la cantidad de ejercicios resueltos por el alumno, indicando con un
valor centinela que no hay más examenes a revisar. Debe mostrar por pantalla
el porcentaje correspondiente a la cantidad de ejercicios resueltos respecto a
la cantidad de ejercicios del examen y una leyenda que indique si aprobó o no.
\end{ejercicio}

% \begin{ejercicio} Resolver el siguiente problema: \\
% Dada una lista de nombres de personas y sus fechas de nacimiento, debe
% indicar que personas son compatibles según su signo zodiacal. En cada paso le
% debe preguntar al usuario si quiere agregar una persona más a la lista, y
% luego mostrar en forma agrupada, los nombres de las personas que son
% compatibles entre sí, según el siguiente criterio: \\
% Dos signos son compatibles si pertencen al mismo elemento: \\
% FUEGO: Aries, Leo, Sagitario. \\
% TIERRA: Tauro, Virgo, Capricornio. \\
% AIRE: Geminis, Libra, Acuario. \\
% AGUA: Cáncer, Escorpio, Piscis.\\
% \end{ejercicio}
