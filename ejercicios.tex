%% 
%% This is file, `ejercicios.tex',
%% generated with the extract package.
%% 
%% Generated on :  2015/04/05,17:26
%% From source  :  principal.tex
%% Using options:  active=true,header=true,generate=ejercicios.tex,copydocumentclass=false,extract-env=ejercicio,extract-cmd=chapter,chapter-nrs={1-11,14-20},
%% 
\documentclass[11pt,a4paper]{article}
\tolerance=5000

% Entrada de texto
\usepackage[spanish]{babel}   % Traduce los textos a castellano
\usepackage[utf8x]{inputenc}  % Permite escribir directamente áéíóúñ
\usepackage{t1enc}            % Agrega caracteres extendidos al font

% Cuestiones de estilo
\usepackage{listings}         % Permite mostrar codigo de forma mas linda
\usepackage{verbatim}         % Permite incluir archivos de texto verbatim

\usepackage{amsmath, amsthm, amssymb} % Se usan para theoremstyle

% Parámetros para los listings de python
\lstset{language=Python,
    numbers=left,
    numberstyle=\tiny,
    numbersep=5pt,
    showstringspaces=false,
    basicstyle=\ttfamily,
    % NOTA: esta sección se podría omitir usando XeTeX.
    extendedchars=true,
    inputencoding=utf8x,
    literate={á}{{\'a}}1
             {é}{{\'e}}1
             {í}{{\'i}}1
             {ó}{{\'o}}1
             {ú}{{\'u}}1
             {ü}{{\"u}}1
             {ñ}{{\~n}}1
             {Á}{{\'A}}1
             {É}{{\'E}}1
             {Í}{{\'I}}1
             {Ó}{{\'O}}1
             {Ú}{{\'U}}1
             {Ü}{{\"U}}1
             {Ñ}{{\~N}}1
             {¿}{{?`}}1
             {¡}{{!`}}1}
%    basicstyle=\ttfamily\normalsize,
%    keywordstyle=\ttfamily\bfseries,
%    identifierstyle=\ttfamily,
%    commentstyle=\ttfamily\itshape,
%    stringstyle=\ttfamily,

% Para poner partes dentro de los ejercicios
\newcounter{partesi}
\newenvironment{partes}
{   \begin{list}{\alph{partesi})}{
        \usecounter{partesi}
        \setlength{\topsep}{0pt}
        \setlength{\itemsep}{0pt}}}
{   \end{list} }

\usepackage{times}
\renewcommand{\baselinestretch}{1.3}

\oddsidemargin 0.0cm \evensidemargin 0.0cm \topmargin 0in
\headheight .3in \headsep .2in \footskip .2in
\setlength{\textwidth}{16cm}
\setlength{\textheight}{22cm}
%\leftmargin 2.5cm
%\rightmargin 2.5cm
\topmargin 0.5 cm

\theoremstyle{definition}
\newtheorem{ejercicio}{Ejercicio}[section]

% Para poder usar los nombres de capítulos como secciones.
% TODO: al omitir capítuos en los ejercicios, las secciones en los ejercicios
% se siguen numerando consecutivamente, y se desacoplan del número de capítulo.
\newcommand{\chapter[2]}{\newpage\section{#2}}

\title{75.40 Algoritmos y Programación I \\
    \textbf{Guía de Ejercicios}}
\date{1er Cuatrimestre 2011}


\begin{document}
 % La guía de estilo para los ejercicios
\maketitle
\thispagestyle{empty}

\newpage

\section*{Recomendaciones al realizar las guías.}

\textbf{Generales:}
\begin{itemize}
\item Sea claro y prolijo. Es muy importante que el código sea lo más claro y legible posible.
\item Es muy importante que los identificadores de funciones y variables sean coherentes. El identificador debe ser suficientemente descriptivo.
\item Ponga una línea en blanco entre las definiciones de función para simplificar la lectura del programa.
\item Las expresiones matemáticas complejas pueden representarse en varios pasos.
\end{itemize}

\textbf{Documentación:}
\begin{itemize}
\item Documente correctamente las funciones y módulos que desarrolle.
\item Documente partes del código cuyo significado pudiera no quedar del todo claro.
\item No documente en exceso, pero tampoco ahorre documentación necesaria. La documentación debe ser breve y concisa.
\end{itemize}

\chapter[Conceptos básicos]{Algunos conceptos básicos}

\begin{ejercicio}
% NOTA: este ejercicio no está incluido en ejercicios.tex.
Correr tres veces el programa \lstinline!cuad! con valores
de entrada (3,5), (3,3) y (5,3) respectivamente. ¿Qué sucede en
cada caso?
\end{ejercicio}

\begin{ejercicio}
% NOTA: este ejercicio no está incluido en ejercicios.tex.
Insertar instrucciones de depuración que permitan ver el valor
asociado con la variable \lstinline!x! en el cuerpo del ciclo \lstinline!for! y
después que se sale de tal ciclo.  Volver a correr tres veces el programa
\lstinline!cuad! con valores de entrada (3,5), (3,3) y (5,3) respectivamente, y
explicar lo que sucede.
\end{ejercicio}

\begin{ejercicio}
% NOTA: este ejercicio no está incluido en ejercicios.tex.
La salida del programa \lstinline!cuad! es poco
informativa. Escribir un programa \lstinline!nom_cuad! que ponga el
número junto a su cuadrado. Ejecutar el programa nuevo.
\end{ejercicio}

\begin{ejercicio}
% NOTA: este ejercicio no está incluido en ejercicios.tex.
Si la salida sigue siendo poco informativa seguir mejorándola hasta
que sea lo suficientemente clara.
\end{ejercicio}

\begin{ejercicio}
Escribir un programa que pregunte al usuario:
\begin{partes}
  \item su nombre, y luego lo salude.
  \item dos números, y luego muestre el producto.
\end{partes}
\end{ejercicio}

\begin{ejercicio} Implementar algoritmos que permitan:
\begin{partes}
 \item Calcular el perímetro y área de un rectángulo dada su base y su altura.
 \item Calcular el perímetro y área de un círculo dado su radio.
 \item Calcular el volumen de una esfera dado su radio.
 \item Calcular el área de un rectángulo (alineado con los ejes x e y) dadas sus coordenadas x1,x2,y1,y2.
 \item Dados los catetos de un triángulo rectángulo, calcular su hipotenusa.
\end{partes}
\end{ejercicio}

\begin{ejercicio}
Mostrar el resultado de ejecutar estos bloques de código en el
intérprete de python:
\begin{partes}
\item \begin{verbatim}
>>> for i in range(5):
        print i * i
\end{verbatim}
\item \begin{verbatim}
>>> for i in range(2,6):
        print i, 2**i
\end{verbatim}
\item \begin{verbatim}
>>> for d in [3, 1, 4, 1, 5]:
        print d,
\end{verbatim}
\end{partes}
\end{ejercicio}

\begin{ejercicio} Implementar algoritmos que resuelvan los siguientes
problemas:
\begin{partes}
  \item Dados dos números, indicar la suma, resta, división y multiplicación
  de ambos.
  \item Dado un número entero N, imprimir su tabla de multiplicar.
  \item Dado un número entero N, imprimir su factorial.
\end{partes}
\end{ejercicio}

\begin{ejercicio}
Escribir un programa que le pida una palabra al usuario, para luego
imprimirla 1000 veces, con espacios intermedios.
\end{ejercicio}

\chapter{Programas sencillos}
En los ejercicios a continuación, utilizar los conceptos de análisis,
especificación y diseño antes de realizar la implementación.

\begin{ejercicio} Ciclos definidos
% NOTA: este ejercicio no está incluido en ejercicios.tex.
\begin{partes}
\item Escribir un ciclo definido para imprimir por pantalla
todos los números entre 10 y 20.
\item Escribir un ciclo definido que salude por pantalla a
sus cinco mejores amigos/as.
\item Escribir un programa que use un ciclo definido con
rango numérico, que pregunte los nombres de sus cinco mejores
amigos/as, y los salude.
\item Escribir un programa que use un ciclo definido con
rango numérico, que pregunte los nombres de sus seis mejores
amigos/as, y los salude.
\item Escribir un programa que use un ciclo definido con
rango numérico, que averigue a cuántos amigos quieren saludar, les
pregunte los nombres de esos amigos/as, y los salude.
\end{partes}
\end{ejercicio}

\begin{ejercicio}
Escribir un programa que le pregunte al usuario una cantidad de pesos,
una tasa de interés y un número de años y muestre como resultado el monto
final a obtener.  La fórmula a utilizar es:
\begin{displaymath}
C_n = C \times (1+\frac{x}{100})^n
\end{displaymath}
Donde $C$ es el capital inicial, $x$ es la tasa de interés y $n$ es el
número de años a calcular.
\end{ejercicio}

\begin{ejercicio}
Escribir un programa que convierta un valor dado en grados Fahrenheit a
grados Celsius.  Recordar que la fórmula para la conversión es:
$F = \frac{9}{5}C+32$
\end{ejercicio}

\begin{ejercicio}
Utilice el programa anterior para generar una tabla de conversión de
temperaturas, desde 0 °F hasta 120 °F, de 10 en 10.
\end{ejercicio}

\begin{ejercicio}
Escribir un programa que imprima todos los números pares entre dos números
que se le pidan al usuario.
\end{ejercicio}

\begin{ejercicio}
Escribir un programa que reciba un número $n$ por
parámetro e imprima los primeros $n$ números triangulares, junto con su
índice. Los números triangulares se obtienen mediante la suma de los números
naturales desde $1$ hasta $n$.  Es decir, si se piden los primeros 5
números triangulares, el programa debe imprimir:

\begin{verbatim}
1 - 1
2 - 3
3 - 6
4 - 10
5 - 15
\end{verbatim}

{\bf Nota}: hacerlo usando y sin usar la ecuación $\sum_{i=1}^n i = n*(n+1)/2$.
¿Cuál realiza más operaciones?
\end{ejercicio}

\begin{ejercicio}
Escribir un programa que tome una cantidad $m$ de valores ingresados
por el usuario, a cada uno le calcule el factorial e imprima el resultado
junto con el número de orden correspondiente.
\end{ejercicio}

\begin{ejercicio}
Escribir un programa que imprima por pantalla todas las fichas de dominó, de
una por línea y sin repetir.
\end{ejercicio}

\begin{ejercicio}
Modificar el programa anterior para que pueda generar fichas de un juego
que puede tener números de 0 a $n$.
\end{ejercicio}

\chapter{Funciones}

\begin{ejercicio} Escribir dos funciones que permitan calcular:
\begin{partes}
    \item La duración en segundos de un intervalo dado en horas, minutos y segundos.
    \item La duración en horas, minutos y segundos de un intervalo dado en segundos.
\end{partes}
\end{ejercicio}

\begin{ejercicio}
Usando las funciones del ejercicio anterior, escribir un programa que pida al
usuario dos intervalos expresados en horas, minutos y segundos, sume sus
duraciones, y muestre por pantalla la duración total en horas, minutos y segundos.
\end{ejercicio}

\begin{ejercicio}
Escribir una función que, dados cuatro números, devuelva el mayor
producto de dos de ellos. Por ejemplo, si recibe los números 1, 5, -2,
-4 debe devolver 8, que es el producto más grande que se puede obtener
entre ellos ($8 = -2 \times -4$).
\end{ejercicio}

\begin{ejercicio}
{\bf Área de un triángulo en base a sus puntos}
\begin{partes}

    \item Escribir una función que dado un vector al origen (definido por sus
 coordenadas \verb!x,y!), devuelva la norma del vector, dada por
 $||\vec{(x,y)}||=\sqrt{x^2+y^2}$

    \item Escribir una función que dados dos puntos en el plano (\verb!x1,y1! y
 \verb!x2,y2!), devuelva la resta de ambos (debe devolver un par de
 valores).

    \item Utilizando las funciones anteriores, escribir una función que dados dos
 puntos en el plano (\verb!x1,y1! y \verb!x2,y2!), devuelva la distancia
 entre ambos.

    \item Escribir una función que reciba un vector al origen (definido por sus
 coordenadas \verb!x,y!), y devuelva el vector normalizado correspondiente (debe
 devolver un par de valores \verb!x',y'!).

    \item Utilizando las funciones anteriores (b y d), escribir una función que
 dados dos puntos en el plano (\verb!x1,y1! y \verb!x2,y2!), devuelva el
 vector dirección unitario correspondiente a la recta que los une.

    \item Escribir una función que reciba un punto (\verb!x,y!), una dirección
 unitaria de una recta (\verb!dx,dy!) y un punto perteneciente a esa recta
 (\verb!cx,cy!) y devuelva la proyección del punto sobre la recta. \\
 {\bf Diseño del algoritmo}:
 \begin{enumerate}
     \setlength{\itemsep}{0pt}
     \setlength{\parsep}{0pt}
     \item Al punto a proyectar (\verb!x,y!) restarle el punto de la recta
 (\verb!cx,cy!)
     \item Obtener la matriz de proyección $P$, dada por:  \\
 $p_{11} = d_x^2$,  $p_{12} = p_{21} = d_x*d_y$, $p_{22} = d_y^2$.
     \item Multiplicar la matriz $P$ por el punto obtenido en el paso 1: \\
 $r_x = p_{11} * x + p_12 * y$, $r_y = p_{21} * x + p_{22} * y$.
     \item Al resultado obtenido sumar el punto restado en el paso 1, y
 devolverlo.
 \end{enumerate}

    \item Escribir una función que calcule el área de un triángulo a partir de
 su base y su altura.

    \item Utilizando las funciones anteriores escribir una función que reciba
 tres puntos en el plano (\verb!x1,y1!, \verb!x2,y2! y \verb!x3,y3!) y
 devuelva el área del triángulo correspondiente.
\end{partes}
\end{ejercicio}

\chapter{Decisiones}

\begin{ejercicio} Escribir dos funciones que resuelvan los siguientes problemas:
\begin{partes}
    \item Dado un número entero $n$, indicar si es par o no.
    \item Dado un número entero $n$, indicar si es primo o no.
\end{partes}
\end{ejercicio}

\begin{ejercicio}
Escribir una implementación propia de la función \verb!abs!, que devuelva
el valor absoluto de cualquier valor que reciba.
\end{ejercicio}

\begin{ejercicio}
Escribir una función que reciba por parámetro una dimensión $n$, e imprima
la matriz identidad correspondiente a esa dimensión.
\end{ejercicio}

\begin{ejercicio}
Escribir funciones que permitan encontrar:
\begin{partes}

    \item  El máximo o mínimo de un polinomio de segundo grado (dados los
coeficientes \verb!a!, \verb!b! y \verb!c!), indicando si es un máximo o un
mínimo.

    \item Las raíces (reales o complejas) de un polinomio de segundo grado. \\
{\bf Nota}: validar que las operaciones puedan efectuarse antes de
realizarlas (no dividir por cero, ni calcular la raíz de un número negativo).

    \item La intersección de dos rectas (dadas las pendientes y ordenada
 al origen de cada recta). \\
{\bf Nota}: validar que no sean dos rectas con la misma pendiente, antes de
efectuar la operación.
\end{partes}
\end{ejercicio}

\begin{ejercicio} Escribir funciones que resuelvan los siguientes problemas:
\begin{partes}
    \item Dado un año indicar si es bisiesto. \\
{\bf Nota}: un año es bisiesto si es un número divisible por 4, pero no si es
divisible por 100, excepto que también sea divisible por 400.

    \item Dado un mes, devolver la cantidad de días correspondientes.

    \item Dada una fecha (día, mes, año), indicar si es válida o no.

    \item Dada una fecha, indicar los días que faltan hasta fin de mes.

    \item Dada una fecha, indicar los días que faltan hasta fin de año.

    \item Dada una fecha, indicar la cantidad de días transcurridos en ese año
hasta esa fecha.

    \item Dadas dos fechas (día1, mes1, año1, día2, mes2, año2), indicar el
tiempo transcurrido entre ambas, en años, meses y días.
\end{partes}
{\bf Nota}: en todos los casos, invocar las funciones escritas previamente
cuando sea posible.
\end{ejercicio}

\begin{ejercicio}
Suponiendo que el primer día del año fue lunes, escribir una función
que reciba un número con el día del año (de 1 a 366) y devuelva el día
de la semana que le toca. Por ejemplo: si recibe '3' debe devolver
'miércoles', si recibe '9' debe devolver 'martes'.
\end{ejercicio}

\begin{ejercicio}
Escribir un programa que reciba como entrada un entero representando un año
(p. ej. 751, 1999, o 2158), y muestre por pantalla el mismo año escrito en
números romanos.
\end{ejercicio}

\begin{ejercicio}
Programa de astrología: el usuario debe ingresar el día y mes de su cumpleaños
y el programa le debe decir a qué signo corresponde. \\
{\bf Nota}: \\
Aries: 21 de marzo al 20 de abril. \\
Tauro: 21 de abril al 20 de mayo. \\
Geminis: 21 de mayo al 21 de junio. \\
Cancer: 22 de junio al 23 de julio. \\
Leo: 24 de julio al 23 de agosto. \\
Virgo: 24 de agosto al 23 de septiembre. \\
Libra: 24 de septiembre al 22 de octubre. \\
Escorpio: 23 de octubre al 22 de noviembre. \\
Sagitario: 23 de noviembre al 21 de diciembre. \\
Capricornio: 22 de diciembre al 20 de enero. \\
Acuario: 21 de enero al 19 de febrero. \\
Piscis: 20 de febrero al 20 de marzo. \\
\end{ejercicio}

\chapter{Más sobre ciclos}

\begin{ejercicio}
Escribir un programa que reciba una a una las notas del usuario,
preguntando a cada paso si desea ingresar más notas, e imprimiendo el
promedio correspondiente.
\end{ejercicio}

\begin{ejercicio}
Escribir una función que reciba un número entero $k$ e imprima su
descomposición en factores primos.
\end{ejercicio}

\begin{ejercicio}
{\bf Manejo de contraseñas}
\begin{partes}
    \item Escribir un programa que contenga una contraseña inventada, que le
pregunte al usuario la contraseña, y no le permita continuar hasta que la
haya ingresado correctamente.
    \item Modificar el programa anterior para que solamente permita una
cantidad fija de intentos.
    \item Modificar el programa anterior para que después de cada intento
agregue una pausa cada vez mayor, utilizando la función \verb!sleep! del
módulo \verb!time!.
    \item Modificar el programa anterior para que sea una función que devuelva
si el usuario ingresó o no la contraseña correctamente, mediante un valor
booleano (True o False).
\end{partes}
\end{ejercicio}

\begin{ejercicio}
Utilizando la función \verb!randrange! del módulo \verb!random!,
escribir un programa que obtenga un número aleatorio secreto, y luego
permita al usuario ingresar números y le indique sin son menores o mayores
que el número a adivinar, hasta que el usuario ingrese el número correcto.
\end{ejercicio}

\begin{ejercicio}
{\bf Algoritmo de Euclides}
\begin{partes}
    \item Implementar en python el algoritmo de Euclides para calcular el máximo
común divisor de dos números $n$ y $m$, dado por los siguientes pasos.
    \begin{enumerate}
        \item Teniendo $n$ y $m$, se obtiene $r$, el resto de la
división entera de $m/n$.
        \item Si $r$ es cero, $n$ es el mcd de los valores iniciales.
        \item Se reemplaza $m \leftarrow n$, $n \leftarrow r$, y se vuelve al
primer paso.
    \end{enumerate}
    \item Hacer un seguimiento del algoritmo implementado para los siguientes
pares de números: (15,9); (9,15); (10,8); (12,6).
\end{partes}
\end{ejercicio}

\begin{ejercicio}
{\bf Potencias de dos.}
\begin{partes}
    \item Escribir una función \verb!es_potencia_de_dos! que reciba como parámetro
un número natural, y devuelva \verb!True! si el número es una potencia de 2,
y \verb!False! en caso contrario.
    \item Escribir una función que, dados dos números naturales pasados como
parámetros, devuelva la suma de todas las potencias de 2 que hay en el
rango formado por esos números (0 si no hay ninguna potencia de 2 entre los
dos). Utilizar la función \verb!es_potencia_de_dos!, descripta en el
punto anterior.
\end{partes}
\end{ejercicio}

\begin{ejercicio}
{\bf Números perfectos y números amigos}
\begin{partes}
    \item Escribir una función que devuelva la suma de todos los divisores de
un número $n$, sin incluirlo.
    \item Usando la función anterior, escribir una función que imprima los
primeros $m$ números tales que la suma de sus divisores sea igual a sí
mismo (es decir los primeros $m$ números {\it perfectos}).
    \item Usando la primera función, escribir una función que imprima las
primeras $m$ parejas de números ($a$,$b$), tales que la suma de los
divisores de $a$ es igual a $b$ y la suma de los divisores de $b$ es igual
a $a$ (es decir las primeras $m$ parejas de números {\it amigos}).
    \item Proponer optimizaciones a las funciones anteriores para disminuir el
tiempo de ejecución.
\end{partes}
\end{ejercicio}

\begin{ejercicio}
Escribir un programa que le pida al usuario que ingrese una sucesión
de números naturales (primero uno, luego otro, y así hasta que el
usuario ingrese '-1' como condición de salida). Al final, el programa
debe imprimir cuántos números fueron ingresados, la suma total de los
valores y el promedio.
\end{ejercicio}

\begin{ejercicio}
Escribir una función que reciba dos números como parámetros, y
devuelva cuántos múltiplos del primero hay, que sean menores que el
segundo.
\begin{partes}
    \item Implementarla utilizando un ciclo \verb!for!, desde el primer número
hasta el segundo.
    \item Implementarla utilizando un ciclo \verb!while!, que multiplique el primer
número hasta que sea mayor que el segundo.
    \item Comparar ambas implementaciones: ¿Cuál es más clara? ¿Cuál realiza menos
operaciones?
\end{partes}
\end{ejercicio}

\begin{ejercicio}
Escribir una función que reciba un número natural e imprima todos
los números primos que hay hasta ese número.
\end{ejercicio}

\begin{ejercicio}
Escribir una función que reciba un dígito y un número natural, y
decida numéricamente si el dígito se encuentra en la notación decimal
del segundo.
\end{ejercicio}

\begin{ejercicio}
Escribir una función que dada la cantidad de ejercicios de un examen,
y el porcentaje necesario de ejercicios bien resueltos necesario para aprobar
dicho examen, revise un grupo de examenes. Para ello, en cada paso debe
preguntar la cantidad de ejercicios resueltos por el alumno, indicando con un
valor centinela que no hay más examenes a revisar. Debe mostrar por pantalla
el porcentaje correspondiente a la cantidad de ejercicios resueltos respecto a
la cantidad de ejercicios del examen y una leyenda que indique si aprobó o no.
\end{ejercicio}

\chapter{Cadenas de caracteres}

\begin{ejercicio}
Escribir funciones que dada una cadena de caracteres:
\begin{partes}
\item Imprima los dos primeros caracteres.
\item Imprima los tres últimos caracteres.
\item Imprima dicha cadena cada dos caracteres. Ej.: \texttt{'recta'} debería
imprimir \texttt{'rca'}
\item Dicha cadena en sentido inverso. Ej.: \texttt{'hola mundo!'} debe
imprimir \texttt{'!odnum aloh'}
\item Imprima la cadena en un sentido y en sentido inverso. Ej:
\texttt{'reflejo'} imprime \texttt{'reflejoojelfer'}.
\end{partes}
\end{ejercicio}

\begin{ejercicio}
Escribir funciones que dada una cadena y un caracter:
\begin{partes}
\item Inserte el caracter entre cada letra de la cadena. Ej: \texttt{'separar'}
y \texttt{','} debería devolver \texttt{'s,e,p,a,r,a,r'}
\item Reemplace todos los espacios por el caracter. Ej: \texttt{'mi archivo de
texto.txt'} y \texttt{'\_'} debería devolver
\texttt{'mi\_archivo\_de\_texto.txt'}
\item Reemplace todos los dígitos en la cadena por el caracter. Ej: \texttt{'su
clave es: 1540'} y \texttt{'X'} debería devolver \texttt{'su clave es: XXXX'}
\item Inserte el caracter cada 3 dígitos en la cadena. Ej.
\texttt{'2552552550'} y \texttt{'.'} debería devolver \texttt{'255.255.255.0'}
\end{partes}
\end{ejercicio}

\begin{ejercicio}
Modificar las funciones anteriores, para que reciban un parámetro que indique
la cantidad máxima de reemplazos o inserciones a realizar.
\end{ejercicio}

\begin{ejercicio}
Escribir una función que reciba una cadena que contiene un largo número entero y
devuelva una cadena con el número y las separaciones de miles. Por ejemplo, si
recibe \texttt{'1234567890'}, debe devolver \texttt{'1.234.567.890'}.
\end{ejercicio}

\begin{ejercicio}
Escribir una función que dada una cadena de caracteres, devuelva:
\begin{partes}
\item La primera letra de cada palabra. Por ejemplo, si recibe
\texttt{'Universal Serial Bus'} debe devolver \texttt{'USB'}.
\item Dicha cadena con la primera letra de cada palabra en mayúsculas. Por
ejemplo, si recibe \texttt{'república argentina'} debe devolver
\texttt{'República Argentina'}.
\item Las palabras que comiencen con la letra 'A'. Por ejemplo, si recibe
\texttt{'Antes de ayer'} debe devolver \texttt{'Antes ayer'}
\end{partes}
\end{ejercicio}

\begin{ejercicio}
Escribir funciones que dada una cadena de caracteres:
\begin{partes}
\item Devuelva solamente las letras consonantes. Por ejemplo, si recibe
\texttt{'algoritmos'} o \texttt{'logaritmos'} debe devolver \texttt{'lgrtms'}.
\item Devuelva solamente las letras vocales. Por ejemplo, si recibe \texttt{'sin
consonantes'} debe devolver \texttt{'i ooae'}.
\item Reemplace cada vocal por su siguiente vocal. Por ejemplo, si recibe
\texttt{'vestuario'} debe devolver \texttt{'vistaerou'}.
\item Indique si se trata de un palíndromo. Por ejemplo, \texttt{'anita
lava la tina'} es un palíndromo (se lee igual de izquierda a derecha que de
derecha a izquierda).
\end{partes}
\end{ejercicio}

\begin{ejercicio}
Escribir funciones que dadas dos cadenas de caracteres:
\begin{partes}
\item Indique si la segunda cadena es una subcadena de la primera. Por ejemplo,
\texttt{'cadena'} es una subcadena de \texttt{'subcadena'}.
\item Devuelva la que sea anterior en orden alfábetico. Por ejemplo, si recibe
\texttt{'kde'} y \texttt{'gnome'} debe devolver \texttt{'gnome'}.
\end{partes}
\end{ejercicio}

\begin{ejercicio}
Escribir una función que reciba una cadena de unos y ceros (es decir, un
número en representación binaria) y devuelva el valor decimal
correspondiente.
\end{ejercicio}

\chapter{Tuplas y listas}

\begin{ejercicio}
Escribir una función que reciba una tupla de elementos e indique si se
encuentran ordenados de menor a mayor o no.
\end{ejercicio}

\begin{ejercicio}
{\bf Dominó.}
\begin{partes}
\item Escribir una función que indique si dos fichas de dominó {\it
encajan} o no. Las fichas son recibidas en dos tuplas, por ejemplo:
\verb!(3,4)! y \verb!(5,4)!
\item Escribir una función que indique si dos fichas de dominó {\it
encajan} o no. Las fichas son recibidas en una cadena, por ejemplo:
\verb!3-4 2-5!. {\bf Nota:} utilizar la función \verb!split! de las cadenas.
\end{partes}
\end{ejercicio}

\begin{ejercicio}{\bf Campaña electoral}
\begin{partes}
\item Escribir una función que reciba una tupla con nombres, y para cada
nombre imprima el mensaje {\it Estimado <nombre>, vote por mí.}
\item Escribir una función que reciba una tupla con nombres, una posición
de origen \verb!p!y una cantidad \verb!n!, e imprima el mensaje anterior
para los \verb!n! nombres que se encuentran a partir de la posición
\verb!p!.
\item Modificar las funciones anteriores para que tengan en cuenta el
género del destinatario, para ello, deberán recibir una tupla de tuplas,
conteniendo el nombre y el género.
\end{partes}
\end{ejercicio}

\begin{ejercicio}
{\bf Vectores}
\begin{partes}
\item Escribir una función que reciba dos vectores y devuelva su producto
escalar.
\item Escribir una función que reciba dos vectores y devuelva si son o no
ortogonales.
\item Escribir una función que reciba dos vectores y devuelva si son
paralelos o no.
\item Escribir una función que reciba un vector y devuelva su norma.
\end{partes}
\end{ejercicio}

\begin{ejercicio}
Dada una lista de números enteros, escribir una función que:
\begin{partes}
\item Devuelva una lista con todos los que sean primos.
\item Devuelva la sumatoria y el promedio de los valores.
\item Devuelva una lista con el factorial de cada uno de esos números.
\end{partes}
\end{ejercicio}

\begin{ejercicio}
Dada una lista de números enteros y un entero k, escribir una función que:
\begin{partes}
\item Devuelva tres listas, una con los menores, otra con los mayores y
otra con los iguales a k.
\item Devuelva una lista con aquellos que son múltiplos de k.
\end{partes}
\end{ejercicio}

\begin{ejercicio}
Escribir una función que reciba una lista de tuplas (Apellido,
Nombre, Inicial\_segundo\_nombre) y devuelva una lista de cadenas
donde cada una contenga primero el nombre, luego la inicial con un punto, y luego el
apellido.
\end{ejercicio}

\begin{ejercicio}{ \bf Inversión de listas}
\begin{partes}
\item Realizar una función que, dada una lista, devuelva una nueva lista cuyo
contenido sea igual a la original pero invertida. Así, dada la lista
\lstinline!['Di', 'buen', 'día', 'a', 'papa']!, deberá devolver
\lstinline!['papa', 'a', 'día', 'buen', 'Di']!.

\item Realizar otra función que invierta la lista, pero en lugar de devolver
una nueva, modifique la lista dada para invertirla, {\bf si}  usar listas
auxiliares.
\end{partes}
\end{ejercicio}

\begin{ejercicio}
Escribir una función "empaquetar" para una lista, donde
epaquetar significa indicar la repetición de valores consecutivos
mediante una tupla (valor, cantidad de repeticiones). Por ejemplo,
\verb!empaquetar ([1,1,1,3,5,1,1,3,3])! debe devolver
\verb![(1, 3) , (3, 1) , (5, 1), (1, 2), (3, 2)]!.
\end{ejercicio}

\begin{ejercicio}
{\bf Matrices.}
\begin{partes}
\item Escribir una función que reciba dos matrices y devuelva la suma.
\item Escribir una función que reciba dos matrices y devuelva el producto.
\item Escribir una función que opere sobre una matriz y mediante {\it
eliminación gaussiana} devuelva una matriz triangular superior.
\item Escribir una función que indique si un grupo de vectores, recibidos
mediante una lista, son linealmente independientes o no.
\end{partes}
\end{ejercicio}

\begin{ejercicio}
{\bf Plegado de un texto.} Escribir una función que reciba un texto y una
longitud y devuelva una lista de cadenas de como máximo esa longitud. Las
líneas deben ser cortadas correctamente en los espacios (sin cortar las
palabras).
\end{ejercicio}

\begin{ejercicio}
{\bf Funciones que reciben funciones.}
\begin{partes}
\item Escribir una funcion llamada {\bf map}, que reciba una función y una
lista y devuelva la lista que resulta de aplicar la función recibida a
cada uno de los elementos de la lista recibida.
\item Escribir una función llamada {\bf filter}, que reciba una función y
una lista y devuelva una lista con los elementos de la lista recibida para
los cuales la función recibida devuelve un valor verdadero.
\item ¿En qué ejercicios de esta guía podría haber utilizado estas
funciones?
\end{partes}
\end{ejercicio}

\chapter{Algoritmos de búsqueda}

\begin{ejercicio}
Escribir una función que reciba una lista desordenada y un elemento, que:
\begin{partes}
\item Busque todos los elementos coincidan con el pasado por parámetro y
devuelva la cantidad de coincidencias encontradas.
\item Busque la primera coincidencia del elemento en la lista y devuelva su
posición.
\item Utilizando la función anterior, busque todos los elementos coincidan
con el pasado por parámetro y devuelva una lista con las posiciones.
\end{partes}
\end{ejercicio}

\begin{ejercicio}
Escribir una función que reciba una lista de números no ordenada, que:
\begin{partes}
\item Devuelva el valor máximo.
\item Devuelva una tupla que incluya el valor máximo y su posición.
\item ¿Qué sucede si los elementos son cadenas de caracteres?
\end{partes}
{\bf Nota:} no utilizar \verb!lista.sort()!
\end{ejercicio}

\begin{ejercicio}
{\bf Agenda simplificada} \\
Escribir una función que reciba una cadena a buscar y una lista de tuplas
(nombre\_completo, telefono), y busque dentro de la lista, todas las
entradas que contengan en el nombre completo la cadena recibida (puede
ser el nombre, el apellido o sólo una parte de cualquiera de ellos).
Debe devolver una lista con todas las tuplas encontradas.
\end{ejercicio}

\begin{ejercicio}
{\bf Sistema de facturación simplificado} \\
Se cuenta con una lista ordenada de productos, en la que uno consiste en
una tupla de (identificador, descripción, precio), y una lista de los
productos a facturar, en la que cada uno consiste en una tupla de
(identificador, cantidad). \\
Se desea generar una factura que incluya la cantidad, la descripción, el
precio unitario y el precio total de cada producto comprado, y al final
imprima el total general. \\
Escribir una función que reciba ambas listas e imprima por
pantalla la factura solicitada.
\end{ejercicio}

\begin{ejercicio}
Escribir una función que reciba una lista ordenada y un elemento, si el
elemento se encuentra en la lista, debe encontrar su posición, mediante
búsqueda binaria y devolverlo.  Si no se encuentra, debe agregarlo a la
lista en la posición correcta y devolver esa nueva posición. (No utilizar
\verb!lista.sort()!.)
\end{ejercicio}

\chapter{Diccionarios}

\begin{ejercicio}
Escribir una función que reciba una lista de tuplas, y que devuelva
un diccionario en donde las claves sean los primeros elementos de las
tuplas, y los valores una lista con los segundos.

Por ejemplo:
\begin{lstlisting}[numbers=none]
l = [ ('Hola', 'don Pepito'), ('Hola', 'don Jose'), ('Buenos', 'días') ]
print tuplas_a_diccionario(l)
\end{lstlisting}

Deberá mostrar: \lstinline!{ 'Hola': ['don Pepito', 'don Jose'], 'Buenos': ['días'] }!
\end{ejercicio}

\begin{ejercicio}
{\bf Diccionarios usados para contar.}
\begin{partes}
  \item Escribir una función que reciba una cadena y devuelva un diccionario con
la cantidad de apariciones de cada palabra en la cadena.  Por ejemplo, si
recibe "Qué lindo día que hace hoy" debe devolver: { 'que': 2, 'lindo': 1,
'día': 1, 'hace': 1, 'hoy': 1}

  \item Escribir una función que cuente la cantidad de apariciones de cada
caracter en una cadena de texto, y los devuelva en un diccionario.

  \item Escribir una función que reciba una cantidad de iteraciones de una tirada
de 2 dados a realizar y devuelva la cantidad de veces que se observa cada valor
de la suma de los dos dados. \\
{\bf Nota}: utilizar el módulo \verb!random! para obtener tiradas aleatorias.
\end{partes}
\end{ejercicio}

\begin{ejercicio}
{\bf Continuación de la agenda.} \\
Escribir un programa que vaya solicitando al usuario que ingrese nombres.
\begin{partes}
  \item Si el nombre se encuentra en la agenda ({\it implementada con un
diccionario}), debe mostrar el teléfono y, opcionalmente, permitir
modificarlo si no es correcto.
  \item Si el nombre no se encuentra, debe permitir ingresar el teléfono
correspondiente.
\end{partes}
El usuario puede utilizar la cadena "*", para salir del programa.
\end{ejercicio}

\begin{ejercicio}
Escribir una función que reciba un texto y para cada caracter presente en el
texto devuelva la cadena más larga en la que se encuentra ese caracter.
\end{ejercicio}

\chapter{Contratos y Mutabilidad}

\begin{ejercicio}
% NOTA: este ejercicio no está incluido en ejercicios.tex.
Realizar la implementación correspondiente a la función \lstinline!sumatoria!.
\end{ejercicio}

\chapter{Manejo de archivos}

\begin{ejercicio}
Escribir un programa, llamado {\bf head} que reciba un archivo y un número
\lstinline!N! e imprima las primeras \lstinline!N! líneas del archivo.
\end{ejercicio}

\begin{ejercicio}
Escribir un programa, llamado {\bf cp.py}, que copie todo el contenido de un
archivo (sea de texto o binario) a otro, de modo que quede exactamente igual.\\
{\bf Nota}: utilizar \lstinline!archivo.read(bytes)! para leer como máximo
una cantidad de bytes.
\end{ejercicio}

\begin{ejercicio}
Escribir un programa, llamado {\bf cut.py}, que dado un archivo de texto, un
delimitador, y una lista de campos, imprima solamente esos campos, separados
por ese delimitador.
\end{ejercicio}

\begin{ejercicio}
Escribir un programa, llamado {\bf wc.py} que reciba un archivo, lo procese e
imprima por pantalla cuántas líneas, cuantas palabras y cuántos caracteres
contiene el archivo.
\end{ejercicio}

\begin{ejercicio}
Escribir un programa, llamado {\bf grep.py} que reciba una expresión y un
archivo e imprima las líneas del archivo que contienen la expresión recibida.
\end{ejercicio}

\begin{ejercicio}
Escribir un programa, llamado {\bf rot13.py} que reciba un archivo de texto de
origen y uno de destino, de modo que para cada línea del archivo origen, se
guarde una línea {\it cifrada} en el archivo destino.  El algoritmo de cifrado
a utilizar será muy sencillo: a cada caracter comprendido entre la a y la z, se
le suma 13 y luego se aplica el módulo 26, para obtener un nuevo caracter.
\end{ejercicio}

\begin{ejercicio} {\bf Persistencia de un diccionario}
\begin{partes}
  \item Escribir una función \lstinline!cargar_datos! que reciba un nombre de
archivo, cuyo contenido tiene el formato \lstinline!clave, valor! y devuelva un
diccionario con el primer campo como clave y el segundo como diccionario.
  \item Escribir una función \lstinline!guardar_datos! que reciba un diccionario
y un nombre de archivo, y guarde el contenido del diccionario en el archivo,
con el formato \lstinline!clave, valor!.
\end{partes}
\end{ejercicio}

\chapter{Objetos}

\begin{ejercicio}
% NOTA: este ejercicio no está incluido en ejercicios.tex.
Modificar el método \lstinline!__cmp__! de \lstinline!Hotel!
para poder ordenar de menor a mayor las listas de hoteles según el criterio:
primero por ubicación, en orden alfabético y dentro de cada ubicación por
la relación calidad-precio.
\end{ejercicio}

\begin{ejercicio}
% NOTA: este ejercicio no está incluido en ejercicios.tex.
Escribir una clase \lstinline!Caja! para representar cuánto
dinero hay en una caja de un negocio, desglosado por tipo de billete (por
ejemplo, en el quiosco de la esquina hay 5 billetes de 10 pesos, 7 monedas
de 25 centavos y 4 monedas de 10 centavos).

Se tiene que poder comparar cajas por la cantidad de dinero que
hay en cada una, y además ordenar una lista de cajas
de menor a mayor según la cantidad de dinero disponible.
\end{ejercicio}

\begin{ejercicio}
Fracciones
\begin{partes}
    \item Crear una clase \verb!Fraccion!, que cuente con dos atributos:
\verb!dividendo! y \verb!divisor!, que se asignan en el constructor, y se
imprimen como \verb!X/Y! en el método \verb!__str__!.
    \item Crear un método \verb!sumar! que recibe otra fracción y devuelve una
nueva fracción con la suma de ambas.
    \item Crear un método \verb!multiplicar! que recibe otra fracción y
devuelve una nueva fracción con el producto de ambas.
    \item Crear un método \verb!simplificar! que modifica la fracción actual de
forma que los valores del \verb!dividendo! y \verb!divisor! sean los
menores posibles.
\end{partes}
\end{ejercicio}

\begin{ejercicio}
Vectores
\begin{partes}
    \item Crear una clase \verb!Vector!, que en su constructor reciba una lista
de elementos que serán sus coordenadas.  En el método \verb!__str__! se
imprime su contenido con el formato \verb![x,y,z]!
    \item Crear un método \verb!escalar! que reciba un número y devuelva un
nuevo vector, con los elementos multiplicados por ese número.
    \item Crear un método \verb!sumar! que recibe otro vector, verifica si
tienen la misma cantidad de elementos y devuelve un nuevo vector con la
suma de ambos.  Si no tienen la misma cantidad de elementos debe levantar
una excepción.
\end{partes}
\end{ejercicio}

\begin{ejercicio}
Botella y Sacacorchos
\begin{partes}
    \item Escribir una clase {\it Corcho}, que contenga un atributo {\it
bodega} (cadena con el nombre de la bodega).
    \item Escribir una clase {\it Botella} que contenga un atributo {\it
corcho} con una referencia al corcho que la tapa, o \verb!None! si está
destapada.
    \item Escribir una clase {\it Sacacorchos} que tenga un método {\it
destapar} que le reciba una botella, le saque el corcho y se guarde una
referencia al corcho sacado.  Debe lanzar una excepción en el caso en que
la botella ya esté destapada, o si el sacacorchos ya contiene un corcho.
    \item Agregar un método {\it limpiar}, que saque el corcho del sacacorchos,
o lance una excepción en el caso en el que no haya un corcho.
\end{partes}
\end{ejercicio}

\begin{ejercicio}
Modelar una clase {\it Mate} que describa el funcionamiento de la
conocida bebida tradicional local. La clase debe contener como miembros:
\begin{partes}
    \item Un atributo para la cantidad de cebadas restantes hasta que se lava
el mate (representada por un número).
    \item Un atributo para el estado (lleno o vacío).
    \item El constructor debe recibir como parámetro \verb!n!, la cantidad
máxima de cebadas en base a la cantidad de yerba vertida en el recipiente.
    \item Un método \verb!cebar!, que llena el mate con agua. Si se intenta
cebar con el mate lleno, se debe lanzar una excepción que imprima el
mensaje "Cuidado! Te quemaste!"
    \item Un método \verb!beber!, que vacía el mate y le resta una cebada
disponible. Si se intenta beber un mate vacío, se debe lanzar una excepción
que imprima el mensaje "El mate está vacío!"
    \item Es posible seguir cebando y bebiendo el mate aunque no haya cebadas
disponibes. En ese caso la cantidad de cebadas restantes se mantendrá
en 0, y cada vez que se intente beber se debe imprimir un mensaje de
aviso: "Advertencia: el mate está lavado.", pero no se debe lanzar una
excepción.
\end{partes}
\end{ejercicio}

\chapter{Polimorfismo, Herencia y Delegación}
Temas optativos para 2010 y 2011.

\begin{ejercicio}
{\bf Papel, Birome, Marcador}
\begin{partes}
    \item Escribir una clase {\it Papel} que contenga un texto, un método {\it
escribir}, que reciba una cadena para agregar al texto, y el método {\it
\_\_str\_\_} que imprima el contenido del texto.
    \item Escribir una clase {\it Birome} que contenga una cantidad de tinta, y
un método {\it escribir}, que reciba un texto y un papel sobre el cual
escribir. Cada letra escrita debe reducir la cantidad de tinta contenida.
Cuando la tinta se acabe, debe lanzar una excepción.
    \item Escribir una clase {\it Marcador} que herede de Birome, y agregue el
método {\it recargar}, que reciba la cantidad de tinta a agregar.
\end{partes}
\end{ejercicio}

\begin{ejercicio}
Juego de Rol
\begin{partes}
    \item Escribir una clase {\it Personaje} que contenga los atributos {\it
vida}, {\it posicion} y {\it velocidad}, y los métodos {\it
recibir\_ataque}, que reduzca la vida según una cantidad recibida y lance
una excepción si la vida pasa a ser menor o igual que cero, y {\it
mover} que reciba una dirección y se mueva en esa dirección la cantidad
indicada por velocidad.
    \item Escribir una clase {\it Soldado} que herede de Personaje, y agregue
el atributo {\it ataque} y el método {\it atacar}, que reciba otro
personaje, al que le debe hacer el daño indicado por el atributo ataque.
    \item Escribir una clase {\it Campesino} que herede de Personaje, y agregue
el atributo {\it cosecha} y el método {\it cosechar}, que devuelva la
cantidad cosechada.
\end{partes}
\end{ejercicio}

\chapter{Listas enlazadas}

\begin{ejercicio}
Agregar a la clase {\it ListaEnlazada} un método \verb!next! que vaya
devolviendo uno a uno cada elemento de la lista, desde el primero hasta el
último.  Al llegar al final de la lista debe levantar una excepción de la
clase {\it StopIteration}.  Para el correcto funcionamiento de este método, ¿es
necesario agregar un atributo adicional a la clase?
\end{ejercicio}

\begin{ejercicio}
Utilizando el método \verb!next! del ejercicio anterior, redefinir el
método \verb!__str__! de {\it ListaEnlazada}, para que se genere una salida
legible de lo que contiene la lista, similar a las listas de python. \\
{\bf Nota}: este método debe devolver una cadena, no imprimirla por
pantalla.
\end{ejercicio}

\begin{ejercicio}
Agregar a {\it ListaEnlazada} un método \verb!extend! que reciba una {\it
ListaEnlazada} y agregue a la lista actual los elementos que se encuentran
en la lista recibida.
\end{ejercicio}

\begin{ejercicio}
Una {\bf lista circular} es una lista cuyo último nodo está ligado al primero,
de modo que es posible recorrerla infinitamente.  \\
Escribir la clase {\it ListaCircular}, incluyendo los métodos \verb!insert!,
\verb!append!, \verb!remove! y \verb!pop!.
\end{ejercicio}

\begin{ejercicio}
Una {\bf lista doblemente enlazada} es una lista en la cual cada nodo tiene
una referencia al anterior además de al próximo de modo que es posible
recorrerla en ambas direcciones. \\
Escribir la clase {\it ListaDobleEnlazada}, incluyendo los métodos
\verb!insert!, \verb!append!, \verb!remove! y \verb!pop!.
\end{ejercicio}

\begin{ejercicio}
Escribir un método de la clase {\it ListaEnlazada} que invierta el orden
de la lista (es decir, el primer elemento queda como último y
viceversa, y se invierte la dirección de todos los enlaces). \\
{\bf Nota}: operar directamente sobre los elementos de la lista.
\end{ejercicio}

\chapter{Pilas y colas}

\begin{ejercicio}
Escribir una clase {\it TorreDeControl} que modele el trabajo de una torre
de control de un aeropuerto, con una pista de aterrizaje.  La torre trabaja
en dos etapas: {\it reconocimiento} y {\it acción}.
\begin{partes}
    \item Escribir un método \verb!reconocimiento!, que verifique si hay algún
nuevo avión esperando para aterrizar y/o despegar, y de ser así los encole
en la cola correspondiente. Para ello, utilizar \verb!random.randrange(2)!.
    \item Escribir un método \verb!acción!, que haga aterrizar o
bien despegar, al primero de los aviones que esté esperando (los que
esperan para aterrizar tienen prioridad). Debe desencolar el avión de su
cola y devolver la información correspondiente.
    \item Escribir un método \verb!__str__! que imprima el estado actual de
ambas colas.
    \item Escribir un programa que inicialice la torre de control, y luego llame
continuamente a los métodos \verb!reconocimiento! y \verb!acción!,
imprimiendo la acción tomada y el estado de la torre de control cada vez.
\end{partes}
\end{ejercicio}

\begin{ejercicio}
Atención a los pacientes de un consultorio médico, con varios doctores.
\begin{partes}
    \item Escribir una clase {\it ColaDePacientes}, con los métodos {\it
nuevo\_paciente}, que reciba el nombre del paciente y lo encole, y un
método {\it proximo\_paciente} que devuelva el primer paciente en la cola y
lo desencole.
    \item Escribir una clase {\it Recepcion}, que contenga un diccionario con
las colas correspondientes a cada doctor o doctora, y los métodos {\it
nuevo\_paciente} que reciba el nombre del paciente y del especialista, y
{\it proximo\_paciente} que reciba el nombre de la persona liberada y
devuelva el próximo paciente en espera.
    \item Escribir un programa que permita al usuario ingresar nuevos pacientes
o indicar que un consultorio se ha liberado y en ese caso imprima el
próximo paciente en espera.
\end{partes}
\end{ejercicio}

\begin{ejercicio}
Juego de Cartas
\begin{partes}
    \item Crear una clase {\it Carta} que contenga un palo y un valor.
    \item Crear una clase {\it PilaDeCartas} que vaya apilando las cartas una
debajo de otra, pero sólo permita apilarlas si son de un número
inmediatamente inferior y de distinto palo. Si se intenta apilar una carta
incorrecta, debe lanzar una excepción.
    \item Agregar el método {\it \_\_str\_\_} a la clase PilaDeCartas, para
imprimir las cartas que se hayan apilado hasta el momento.
\end{partes}
\end{ejercicio}

\chapter{Modelo de ejecución de funciones y recursividad}

\begin{ejercicio}
Escribir una función que reciba un número positivo $n$ y devuelva
la cantidad de dígitos que tiene.
\end{ejercicio}

\begin{ejercicio}
Escribir una función que simule el siguiente experimento:
Se tiene una rata en una jaula con 3 caminos, entre los cuales elige
al azar (cada uno tiene la misma probabilidad), si elige el {\it 1} luego
de 3 minutos vuelve a la jaula, si elige el {\it 2} luego de 5 minutos vuelve a
la jaula, en el caso de elegir el {\it 3} luego de 7 minutos sale de la jaula.
La rata no aprende, siempre elige entre los 3 caminos con la misma probabilidad,
pero quiere su libertad, por lo que recorrerá los caminos hasta salir de la jaula.

La función debe devolver el tiempo que tarda la rata en salir de la jaula.
\end{ejercicio}

\begin{ejercicio}
Escribir una función que reciba 2 enteros {\it n} y {\it b} y devuelva
\verb!True! si {\it n} es potencia de {\it b}.

Ejemplos:
\begin{verbatim}
>>> es_potencia(8,2)
True
>>> es_potencia(64,4)
True
>>> es_potencia(70,10)
False
\end{verbatim}
\end{ejercicio}

\begin{ejercicio}
Escribir una funcion recursiva que reciba como parámetros dos strings {\it a} y
{\it b}, y devuelva una lista con las posiciones en donde se encuentra {\it b}
dentro de {\it a}.

Ejemplo:
\begin{verbatim}
>>> posiciones_de("Un tete a tete con Tete", "te")
[3, 5, 10, 12, 21]
\end{verbatim}
\end{ejercicio}

\begin{ejercicio}
Escribir dos funciones mutualmente recursivas par(n) e impar(n) que
determinen la paridad del numero natural dado, conociendo solo que:
\begin{itemize}
    \item 1 es impar.
    \item Si un número es impar, su antecesor es par; y viceversa.
\end{itemize}
\end{ejercicio}

\begin{ejercicio}
Escribir una función que calcule recursivamente el n-ésimo número
triangular (el número 1 + 2 + 3 + ... + n).
\end{ejercicio}

\begin{ejercicio}
Escribir una función que calcule recursivamente cuántos elementos
hay en una pila, suponiendo que la pila sólo tiene los métodos apilar
y desapilar, y no altere el contenido de la pila.\\
¿Implementarías esta función para un programa real? ¿Por qué?
\end{ejercicio}

\begin{ejercicio}
Escribir una funcion recursiva que encuentre el mayor elemento de una lista.
\end{ejercicio}

\begin{ejercicio}
Escribir una función recursiva para replicar los elementos de una lista
una cantidad n de veces. Por ejemplo,
\verb!replicar ([1, 3, 3, 7], 2) = ([1, 1, 3, 3, 3, 3, 7, 7])!
\end{ejercicio}

\chapter{Ordenar listas}

\begin{ejercicio}
Implementar una función que reciba una lista y devuelva otra lista con los
mismos elementos que la primera, ordenados de mayor a menor mediante el método
de inserción.
\end{ejercicio}

\chapter{Algunos ordenamientos recursivos}

\begin{ejercicio}
Escribir una función \verb!merge_sort_3! que funcione igual que el merge sort
pero en lugar de dividir los valores en dos partes iguales, los divida en tres
(asumir que se cuenta con la función \verb!merge(lista_1, lista_2, lista_3)!).
¿Cómo te parece que se va a comportar este método con respecto al merge sort
original?
\end{ejercicio}

\begin{ejercicio}
Mostrar los pasos del ordenamiento de la lista: 6,0,3,2,5,7,4,1 con
los métodos de inserción y con merge sort. ¿Cuáles son las principales
diferencias entre los métodos? ¿Cuál usarías en qué casos?
\end{ejercicio}

\begin{ejercicio}
Ordenar la lista 6,0,3,2,5,7,4,1 usando el método quicksort. Mostrar
el árbol de recursividad explicando las llamadas que se hacen en cada
paso, y el orden en el que se realizan.
\end{ejercicio}

\end{document}
