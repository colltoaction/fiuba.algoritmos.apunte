\chapter{Contratos y Mutabilidad}

En esta unidad se le dará cierta formalización a algunos temas que habían sido
presentados informalmente, como por ejemplo la documentación de las funciones.

Se formalizarán las condiciones que debe cumplir un algoritmo al comenzar, en
su transcurso, y al terminar, y algunas técnicas para tener en cuenta estas
condiciones.

También se verá una forma de modelar el espacio donde \textit{viven} las
variables.

\section{Pre y Postcondiciones}

Cuando hablamos de \textit{contratos} o \textit{programación por
contratos}, nos referimos a la necesidad de estipular tanto lo que necesita
como lo que devuelve nuestro código.

Las condiciones que deben estar dadas para que el código funcione las llamamos
\emph{precondiciones} y las condiciones sobre el estado en que quedan las
variables y él o los valores de retorno, las llamamos \emph{postcondiciones}.

En definitiva, este concepto es similar al ya mencionado con respecto a la
documentación de funciones, es decir que se debe \emph{documentar cómo deben
ser los parámetros recibidos, cómo va a ser lo que se devuelve, y qué
sucede con los parámetros en caso de ser modificados}.

Esta estipulación es mayormente para que la utilicen otros programadores,
por lo que es particularmente útil cuando se encuentra dentro de la
documentación. En ciertos casos, además, puede quererse que el programa
revise si las condiciones realmente se cumplen y de no ser así, actúe en
consecuencia.

Existen herramientas en algunos lenguajes de programación que facilitan estas
acciones, en el caso de Python, es posible utilizar la instrucción
\lstinline!assert!.

\subsection{Precondiciones}

Las precondiciones son las condiciones que deben cumplir los parámetros que
una función recibe, para que esta se comporte correctamente.

Por ejemplo, en una función división las precondiciones son que los parámetros
son números, y que el divisor sea distinto de 0. Tener una precondición
permite asumir desde el código que no es necesario lidiar con los casos en que
las precondiciones no se cumplen.

\subsection{Postcondiciones}

Las postcodiciones son las condiciones que cumplirá el valor de retorno, y
los parámetros recibidos, en caso de que hayan sido alterados,
siempre que se hayan cumplido las precondiciones.

En el ejemplo anterior, la función división con las precondiciones asignadas,
puede asegurar que devolverá un número correspondiente al cociente solicitado.

\subsection{Aseveraciones}

Tanto las precondiciones como las postcondiciones son \textit{aseveraciones}
(en inglés \textit{assertions}). Es decir, afirmaciones realizadas en un momento
particular de la ejecución sobre el estado computacional. Si llegaran a ser
falsas significaría que hay algún error en el diseño o utilización del algoritmo.

Para comprobar estas afirmaciones desde el código en algunos casos podemos
utilizar la instrucción \lstinline!assert!. Esta instrucción recibe una
condición a verificar (o sea, una expresión booleana)
Si la condición es |True|, la instrucción no hace nada; en caso contrario se
produce un error.

\begin{codigo-python-sn}
>>> assert True
>>> assert False
Traceback (most recent call last):
  File "<stdin>", line 1, in <module>
AssertionError
\end{codigo-python-sn}

Opcionalmente, la instrucción |assert| puede recibir
mensaje de error que mostrará en caso que la condición no se cumpla.

\begin{codigo-python-sn}
>>> n = 0
>>> assert n != 0, "El divisor no puede ser 0"
Traceback (most recent call last):
  File "<stdin>", line 1, in <module>
AssertionError: El divisor no puede ser 0
\end{codigo-python-sn}

\begin{atencion}
Es importante tener en cuenta que \lstinline!assert! está pensado para ser
usado en la etapa de desarrollo. Un programa terminado nunca debería dejar
de funcionar por este tipo de errores.
\end{atencion}

\subsection{Ejemplos}

Usando los ejemplos anteriores, la función \lstinline!division! nos
quedaría de la siguiente forma:

\begin{codigo-python-sn}
def division(dividendo, divisor):
    """Cálculo de la división

    Pre: Recibe dos números, divisor debe ser distinto de 0.
    Post: Devuelve un número real, con el cociente de ambos.
    """
    assert divisor != 0, "El divisor no puede ser 0"
    return dividendo / divisor
\end{codigo-python-sn}

Otro ejemplo, tal vez más interesante, puede ser una función que implemente
una sumatoria ($\sum_{i=inicial}^{final} f(i)$).  En este caso hay que
analizar cuáles van a ser los parámetros que recibirá la función, y las
precondiciones que estos parámetros deberán cumplir.

La función |sumatoria| a escribir necesita de un valor inicial, un valor
final, y una función a la cual llamar en cada paso. Es decir que recibe
tres parámetros.

\begin{codigo-python-sn}
def sumatoria(inicial, final, f):
\end{codigo-python-sn}

Tanto \lstinline!inicial! como \lstinline!final! deben ser números enteros,
y dependiendo de la implementación a realizar o de la especificación
previa, puede ser necesario que \lstinline!final! deba ser mayor o igual a
\lstinline!inicial!.

Con respecto a \lstinline!f!, se trata de una función que será llamada con
un parámetro en cada paso y se requiere poder sumar el resultado, por lo
que debe ser una función que reciba un número y devuelva un número.

La declaración de la función queda, entonces, de la siguiente manera.

\begin{codigo-python-sn}
def sumatoria(inicial, final, f):
    """Calcula la sumatoria desde i=inicial hasta final de f(i)

    Pre: inicial y final son números enteros, f es una función que
         recibe un entero y devuelve un número.
    Post: Se devuelve el valor de la sumatoria de aplicar f a cada
          número comprendido entre inicial y final.
    """
\end{codigo-python-sn}

\begin{ejercicio}
Realizar la implementación correspondiente a la función \lstinline!sumatoria!.
\end{ejercicio}

En definitiva, la documentación de pre y postcondiciones dentro de la documentación
de las funciones es una forma de especificar claramente el comportamiento del
código de forma que quienes lo vayan a utilizar no requieran conocer cómo está
implementado para poder aprovecharlo.

Esto es útil incluso en los casos en los que el programador de las funciones
es el mismo que el que las va a utilizar, ya que permite separar
responsabilidades. Las pre y postcondiciones son, en efecto, un
\textit{contrato} entre el código invocante y el invocado.

\section{Invariantes de ciclo}

% TODO: conseguir frase, la vida es siempre igual, siempre está cambiando.
%\begin{quote}
%``Dadme un punto de apoyo y moveré el mundo'' Arquímedes
%\end{quote}

Los invariantes se refieren a estados o situaciones que no cambian dentro
de un contexto o porción de código.  Hay invariantes de ciclo, que son los
que veremos a continuación, e invariantes de estado, que se verán más
adelante.

El invariante de ciclo permite conocer cómo llegar desde las precondiciones
hasta las postcondiciones. El invariante de ciclo es, entonces, una
aseveración que debe ser verdadera al comienzo de cada iteración.

Por ejemplo, si el problema es ir desde el punto $A$ al punto $B$, las
precondiciones dicen que estamos parados en $A$ y las postcondiciones que
estamos parados en $B$, un invariante podría ser ``estamos en algún punto entre
$A$ y $B$, en el punto más cercano a $B$ que estuvimos hasta ahora.''.

Más específicamente, si analizamos el ciclo para buscar el máximo en una lista
desordenada, la precondición es que la lista contiene elementos que son
comparables y la postcondición es que se devuelve el elemento máximo de la
lista.

\begin{codigo-python-sn}
def maximo(lista):
    "Devuelve el elemento máximo de la lista o None si está vacía."
    if not lista:
        return None
    max_elem = lista[0]
    for elemento in lista:
        if elemento > max_elem:
            max_elem = elemento
    return max_elem
\end{codigo-python-sn}

En este caso, el invariante del ciclo es que \lstinline!max_elem! contiene el
valor máximo de la porción de lista analizada.

Los invariantes son de gran importancia al momento de demostrar que un
algoritmo funciona, pero aún cuando no hagamos una demostración formal es muy
útil tener los invariantes a la vista, ya que de esta forma es más fácil
entender cómo funciona un algoritmo y encontrar posibles errores.

Los invariantes, además, son útiles a la hora de determinar las condiciones
iniciales de un algoritmo, ya que también deben cumplirse para ese caso.  Por
ejemplo, consideremos el algoritmo para obtener la potencia \lstinline!n! de
un número.

\begin{codigo-python-sn}
def potencia(b, n):
    "Devuelve la potencia n del número b, con n entero mayor que 0."
    p = 1
    for i in range(n):
        p *= b
    return p
\end{codigo-python-sn}

En este caso, el invariante del ciclo es que la variable \lstinline!p!
contiene el valor de la potencia correspondiente a esa iteración. Teniendo en
cuenta esta condición, es fácil ver que \lstinline!p! debe comenzar el ciclo
con un valor de 1, ya que ese es el valor correspondiente a $p^0$.

De la misma manera, si la operación que se quiere realizar es sumar todos los
elementos de una lista, el invariante será que una variable \lstinline!suma!
contenga la suma de todos los elementos ya recorridos, por lo que es claro que
este invariante debe ser 0 cuando aún no se haya recorrido ningún elemento.

\begin{codigo-python-sn}
def suma(lista):
    "Devuelve la suma de todos los elementos de la lista."
    suma = 0
    for elemento in lista:
        suma += elemento
    return suma
\end{codigo-python-sn}

% TODO
% \subsection{Invariantes como medida de cuánto falta}

%Dependiendo del problema y las herramientas con las que contemos algunos
%invariantes se pueden medir retomando el ejemplo de ir de A a B, uno podría
%medir la distancia hasta B para esta, pero si para medir la distancia hay que ir hasta B
%y volver deja de tener sentido. O al estar buscando el mínimo en una
%secuencia, cómo hago para comprobar que paso a paso tengo el mínimo de la
%secuencia que ya recorrí sin usar

\subsection{Comprobación de invariantes desde el código}

Cuando la comprobación necesaria para saber si seguimos ``en camino'' es simple,
se la puede tener directamente dentro del código.  Evitando seguir avanzando
con el algoritmo si se produjo un error crítico.

Por ejemplo, en una búsqueda binaria, el elemento a buscar debe ser mayor que
el elemento inicial y menor que el elemento final, de no ser así, no tiene sentido
continuar con la búsqueda.  Es posible, entonces, agregar una instrucción
que compruebe esta condición y de no ser cierta realice alguna acción para
indicar el error, por ejemplo, utilizando la instrucción \lstinline!assert!,
vista anteriormente.

\section{Mutabilidad e Inmutabilidad}

Hasta ahora cada vez que estudiamos un tipo de datos indicamos si son
mutables o inmutables.

Cuando un valor es de un tipo inmutable, como por ejemplo una cadena, es
posible asignar un nuevo valor a esa variable, pero no es posible modificar su
contenido.

\begin{codigo-python-sn}
>>> s = "ejemplo"
>>> s = "otro"
>>> s[2] = "c"
Traceback (most recent call last):
  File "<stdin>", line 1, in <module>
TypeError: 'str' object does not support item assignment
\end{codigo-python-sn}

Esto se debe a que cuando se realiza una nueva asignación, no se modifica la
cadena en sí, sino que la variable \lstinline!s! pasa a \emph{referenciar} a otra cadena.
En cambio, no es posible asignar un nuevo caracter en una posición, ya que
esto implicaría modificar la cadena inmutable.

En el caso de los parámetros mutables, la asignación tiene el mismo
comportamiento, es decir que las variables pasan a apuntar a un nuevo valor.

\begin{codigo-python-sn}
>>> lista1 = [10, 20, 30]
>>> lista2 = lista1
>>> lista1 = [3, 5, 7]
>>> lista1
[3, 5, 7]
>>> lista2
[10, 20, 30]
\end{codigo-python-sn}

Algo importante a tener en cuenta en el caso de las variables de tipo
mutable es que si hay dos o más variables que \textit{referencian} a un mismo
valor, y este valor se modifica, el cambio se verá reflejado en ambas variables.

\begin{codigo-python-sn}
>>> lista1 = [1, 2, 3]
>>> lista2 = lista1
>>> lista2[1] = 5
>>> lista1
[1, 5, 3]
\end{codigo-python-sn}

\begin{sabias_que}
En otros lenguajes, como C o C++, existe un tipo de variable especial
llamado {\it puntero}, que se comporta como una referencia a un valor,
como es el caso de las variables mutables del ejemplo anterior.

En Python no hay punteros como los de C o C++, pero todas las variables son
referencias a una porción de memoria, de modo que cuando se asigna una
variable a otra, lo que se está asignando es la porción de memoria a la que
refieren.  Si esa porción de memoria cambia, el cambio se puede ver en
todas las variables que apuntan a esa porción.
\end{sabias_que}

% TODO: describir en más detalle el modelo de referencia de python ?

\subsection{Parámetros mutables e inmutables}

Las funciones reciben parámetros que pueden ser mutables o inmutables.

Si dentro del cuerpo de la función se modifica uno de estos parámetros para
que \textit{referencie} a otro valor, este cambio no se verá reflejado fuera de la
función.  Si, en cambio, se modifica el \textit{contenido} de alguno de los
parámetros mutables, este cambio \textit{sí} se verá reflejado fuera de la
función.

A continuación un ejemplo en el cual se asigna la variable recibida, a un
nuevo valor.  Esa asignación sólo tiene efecto dentro de la función.

\begin{codigo-python-sn}
>>> def no_cambia_lista(lista):
...     lista = range(len(lista))
...     print("Dentro de la funcion lista =", lista)
...
>>> lista = [10, 20, 30, 40]
>>> no_cambia_lista(lista)
Dentro de la funcion lista = [0, 1, 2, 3]
>>> lista
[10, 20, 30, 40]
\end{codigo-python-sn}

A continuación un ejemplo en el cual se modifica la variable recibida. En este
caso, los cambios realizados tienen efecto tanto dentro como fuera de la
función.

\begin{codigo-python-sn}
>>> def cambia_lista(lista):
...     for i in range(len(lista)):
...         lista[i] = lista[i] ** 3
...
>>> lista = [1, 2, 3, 4]
>>> cambia_lista(lista)
>>> lista
[1, 8, 27, 64]
\end{codigo-python-sn}

\begin{atencion}
En general, se espera que una función que recibe parámetros mutables no los
modifique, ya que si se los modifica se podría perder información valiosa.

En el caso en que por una decisión de diseño o especificación se modifiquen
los parámetros recibidos, esto debe estar claramente documentado, dentro de
las postcondiciones.
\end{atencion}

\section{Resumen}

\begin{itemize}
\item Las \textbf{precondiciones} son las condiciones que deben cumplir los
parámetros recibidos por una función.
\item Las \textbf{postcondiciones} son las condiciones cumplidads por los
resultados que la función devuelve y por los parámetros recibidos, siempre
que las precondiciones hayan sido válidas.
\item Los \textbf{invariantes de ciclo} son las condiciones que deben
cumplirse al comienzo de cada iteración de un ciclo.
\item En el caso en que estas \textbf{aseveraciones} no sean verdaderas, se
deberá a un error en el diseño o utilización del código.
\item En general una función no debe modificar el contenido de sus parámetros,
aún cuando esto sea posible, a menos que sea la funcionalidad explícita de esa
función.
\end{itemize}

\begin{referencia_python}

\begin{sintaxis}{\lstinline!assert condicion[, mensaje]!}
Verifica si la condición es verdadera.  En caso contrario, provoca un error
con el mensaje recibido por parámetro.
\end{sintaxis}

\end{referencia_python}

\section{Apéndice - Acertijo MU}

El acertijo MU\footnote{
\url{http://en.wikipedia.org/wiki/Invariant\_(computer\_science)}} es un buen
ejemplo de un problema lógico donde es útil determinar el invariante.  El
acertijo consiste en buscar si es posible convertir MI a MU, utilizando las
siguientes operaciones.

\begin{enumerate}
\item Si una cadena termina con una I, se le puede agregar una U (xI -> xIU)
\item Cualquier cadena luego de una M puede ser totalmente duplicada (Mx ->
Mxx)
\item Donde haya tres Is consecutivas (III) se las puede reemplazar por una U
(xIIIy -> xUy)
\item Dos Us consecutivas, pueden ser eliminadas (xUUy -> xy)
\end{enumerate}

Para resolver este problema, es posible pasar horas aplicando estas reglas
a distintas cadenas.  Sin embargo, puede ser más fácil encontrar una
afirmación que sea invariante para todas las reglas y que muestre si es o
no posible llegar a obtener MU.

Al analizar las reglas, la forma de deshacerse de las Is es conseguir tener
tres Is consecutivas en la cadena.  La única forma de deshacerse de todas las
Is es que haya un cantidad de Is consecutivas múltiplo de tres.

Es por esto que es interesante considerar la siguiente afirmación como
invariante: el número de Is en la cadena no es múltiplo de tres.

Para que esta afirmación sea invariante al acertijo, para
cada una de las reglas se debe cumplir que: si el invariante era verdadero
antes de aplicar la regla, seguirá siendo verdadero luego de aplicarla.

Para ver si esto es cierto o no, es necesario considerar la aplicación del
invariante para cada una de las reglas.

\begin{enumerate}
\item Se agrega una U, la cantidad de Is no varía, por lo cual se mantiene el
invariante.
\item Se duplica toda la cadena luego de la M, siendo $n$ la cantidad de
Is antes de la duplicación, si $n$ no es múltiplo de 3, $2n$ tampoco lo será.
\item Se reemplazan tres Is por una U.  Al igual que antes, siendo $n$ la
cantidad de Is antes del reemplazo, si $n$ no es múltiplo de 3, $n-3$ tampoco
lo será.
\item Se eliminan Us, la cantidad de Is no varía, por lo cual se mantiene el
invariante.
\end{enumerate}

Todo esto indica claramente que el invariante se mantiene para cada una de las
posibles transformaciones.  Esto significa que sea cual fuere la regla que se
elija, si la cantidad de Is no es un múltiplo de tres antes de aplicarla, no
lo será luego de hacerlo.

Teniendo en cuenta que hay una única I en la cadena inicial MI, y que uno no
es múltiplo de tres, es imposible llegar a MU con estas reglas, ya que MU
tiene cero Is, que sí es múltiplo de tres.

