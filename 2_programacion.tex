% Copyright (C) Rosita Wachenchauzer <rositaw@gmail.com>

% Esta obra está licenciada de forma dual, bajo las licencias Creative
% Commons:
%  * Atribución-Compartir Obras Derivadas Igual 2.5 Argentina
%    http://creativecommons.org/licenses/by-sa/2.5/ar/
%  * Atribución-Compartir Obras Derivadas Igual 3.0 Unported
%    http://creativecommons.org/licenses/by-sa/3.0/deed.es_AR.
%
% A su criterio, puede utilizar una u otra licencia, o las dos.
% Para ver una copia de las licencias, puede visitar los sitios
% mencionados, o enviar una carta a Creative Commons,
% 171 Second Street, Suite 300, San Francisco, California, 94105, USA.

\chapter[Programas sencillos]{Programas sencillos}

En esta unidad empezaremos a resolver problemas sencillos, y a
programarlos en Python.

\section{Construcción de programas}

Cuando nos piden que hagamos un programa debemos seguir una cierta
cantidad de pasos para asegurarnos de que tendremos éxito en la
tarea. La acción irreflexiva (me piden algo, me siento frente a la
computadora y escribo rápidamente y sin pensarlo lo que me parece
que es la solución) no constituye una actitud profesional (e
ingenieril) de resolución de problemas. Toda construcción tiene
que seguir una metodología, un protocolo de desarrollo, dado.

Existen muchas metodologías para construir programas, pero en este
curso aplicaremos una metodología sencilla, que es adecuada para
la construcción de programas pequeños, y que se puede resumir en
los siguientes pasos:

\begin{enumerate}
\item {\bf Analizar el problema.} Entender profundamente {\it cuál} es
el problema que se trata de resolver, incluyendo el contexto en el
cual se usará.

\begin{observacion}
Una vez analizado el problema, asentar el análisis por escrito.
\end{observacion}

\item {\bf Especificar la solución.} Éste es el punto en el cual
se describe {\it qué} debe hacer el programa, sin importar
el cómo. En el caso de los problemas sencillos que abordaremos,
deberemos decidir cuáles son los datos de entrada que se nos
proveen, cuáles son las salidas que debemos producir, y cuál es la
relación entre todos ellos.

\begin{observacion}
Al especificar el problema a resolver, documentar la especificación por
escrito.
\end{observacion}

\item {\bf Diseñar la solución.} Éste es el punto en el cuál
atacamos el {\it cómo} vamos a resolver el problema, cuáles son
los algoritmos y las estructuras de datos que usaremos. Analizamos
posibles variantes, y las decisiones las tomamos usando como dato
de la realidad el contexto en el que se aplicará la solución, y
los costos asociados a cada diseño.

\begin{observacion}
Luego de diseñar la solución, asentar por escrito el diseño, asegurándonos de
que esté completo.
\end{observacion}

\item {\bf Implementar el diseño.} Traducir a un lenguaje de
programación (en nuestro caso, y por el momento, Python) el diseño
que elegimos en el punto anterior.

\begin{observacion}
La implementación también se debe documentar, con comentarios
dentro y fuera del código, al respecto de qué hace el programa, cómo lo hace y
por qué lo hace de esa forma.
\end{observacion}

\item {\bf Probar el programa.} Diseñar un conjunto de pruebas
para probar cada una de sus partes por separado, y también la
correcta integración entre ellas. Utilizar el {\it depurador} como
instrumento para descubir dónde se producen ciertos errores.

\begin{observacion}
Al ejecutar las pruebas, documentar los resultados obtenidos.
\end{observacion}

\item {\bf Mantener el programa.} Realizar los cambios en
respuesta a nuevas demandas.

\begin{observacion}
Cuando se realicen cambios, es necesario documentar el análisis,
la especificación, el diseño, la implementación y las pruebas que surjan para
llevar estos cambios a cabo.
\end{observacion}

\end{enumerate}

\section{Realizando un programa sencillo}

Al leer un artículo en una revista norteamericana que contiene información de
longitudes expresadas en millas, pies y pulgadas, queremos poder convertir esas
distancias de modo que sean fáciles de entender.  Para ello, decidimos escribir
un programa que convierta las longitudes del sistema inglés al sistema métrico
decimal.

Antes de comenzar a programar, utilizamos la guía de la sección anterior, para
analizar, especificar, diseñar, implementar y probar el problema.

\begin{enumerate}
\item {\bf Análisis del problema.} En este caso el problema es
sencillo: nos dan un valor expresado en millas, pies y pulgadas y
queremos transformarlo en un valor en el sistema métrico decimal.
Sin embargo hay varias respuestas posibles, porque no hemos fijado
en qué unidad queremos el resultado. Supongamos que decidimos que
queremos expresar todo en metros.

\item {\bf Especificación.} Debemos establecer la relación entre
los datos de entrada y los datos de salida. Ante todo debemos
averiguar los valores para la conversión de las unidades básicas.
Buscando en Internet encontramos la siguiente tabla:

\begin{itemize}
\item 1 milla = 1.609344 km
\item 1 pie = 30.48 cm
\item 1 pulgada = 2.54 cm
\end{itemize}

\begin{atencion}
A lo largo de todo el curso usaremos punto decimal,
en lugar de coma decimal, para representar valores no enteros,
dado que esa es la notación que utiliza Python.
\end{atencion}

La tabla obtenida no traduce las longitudes a metros. La manipulamos para
llevar todo a metros:

\begin{itemize}
\item 1 milla = 1609.344 m
\item 1 pie = 0.3048 m
\item 1 pulgada = 0.0254 m
\end{itemize}

Si una longitud se expresa como $L$ millas, $F$ pies y $P$ pulgadas, su
conversión a metros se calculará como
$M = 1609.344 * L + 0.3048 * F + 0.0254 * P.$

Hemos especificado el problema. Pasamos entonces a la próxima etapa.

\item {\bf Diseño.} La estructura de este programa es sencilla:
leer los datos de entrada, calcular la solución, mostrar el
resultado, o {\it Entrada-Cálculo-Salida}.

Antes de escribir el programa, escribiremos en {\it pseudocódigo}
(un castellano preciso que se usa para describir lo que hace un
programa) una descripción del mismo:

\begin{verbatim}
Leer cuántas millas tiene la longitud dada
 (y referenciarlo con la variable millas)

Leer cuántos pies tiene la longitud dada
 (y referenciarlo con la variable pies)

Leer cuántas pulgadas tiene la longitud dada
 (y referenciarlo con la variable pulgadas)

Calcular metros = 1609.344 * millas +
    0.3048 * pies + 0.0254 * pulgadas

Mostrar por pantalla la variable metros
\end{verbatim}

\item {\bf Implementación.} Ahora estamos en condiciones de
traducir este pseudocódigo a un programa en lenguaje Python:

\begin{codigo}{ametrico.py}{Convierte medidas inglesas a sistema metrico}
\begin{codigo-python}
def main():
      print "Convierte medidas inglesas a sistema metrico"
      millas = input("Cuántas millas?: ")
      pies = input("Y cuántos pies?: ")
      pulgadas = input("Y cuántas pulgadas?: ")

      metros = 1609.344 * millas + 0.3048 * pies + 0.0254 * pulgadas
      print "La longitud es de ", metros, " metros"
main()
\end{codigo-python}
\end{codigo}

\item {\bf Prueba.} Probaremos el programa para valores para los que conocemos
la solución:

\begin{itemize}
\item 1 milla, 0 pies, 0 pulgadas.
\item 0 millas, 1 pie, 0 pulgada.
\item 0 millas, 0 pies, 1 pulgada.
\end{itemize}

La prueba la documentaremos con la sesión de Python
correspondiente a las tres invocaciones a \lstinline!ametrico.py!.
\end{enumerate}

En la sección anterior hicimos hincapié en la necesidad de
documentar todo el proceso de desarrollo. En este ejemplo la
documentación completa del proceso lo constituye todo lo escrito
en esta sección.

\begin{atencion}
Al entregar un ejercicio, se deberá presentar el desarrollo completo con
todas las etapas, desde el análisis hasta las pruebas (y el mantenimiento, si
hubo cambios).
\end{atencion}

\section {Piezas de un programa Python}
Para poder empezar a programar en Python es necesario que conocer los elementos
que constituyen un programa en dicho lenguaje y las reglas para construirlos.

\begin{observacion}
Cuando empezamos a hablar en un idioma extranjero es posible que nos entiendan
pese a que cometamos errores. No sucede lo mismo con los lenguajes de
programación: la computadora no nos entenderá si nos desviamos un poco de
alguna de las reglas.
\end{observacion}

\subsection{Nombres}
Ya hemos visto que se usan nombres para denominar a los programas
(\lstinline!ametrico!) y para denominar a las funciones dentro de un
módulo (\lstinline!main!). Cuando queremos dar nombres a valores usamos
variables (\lstinline!millas!, \lstinline!pies!, \lstinline!pulgadas!,
\lstinline!metros!). Todos esos nombres se llaman {\it identificadores}
y Python tiene reglas sobre qué es un identificador válido y qué
no lo es.

Un identificador comienza con una letra o con guión bajo (\_) y
luego sigue con una secuencia de letras, números y guiones bajos.
Los espacios no están permitidos dentro de los identificadores.

Los siguientes son todos identificadores válidos de Python:

\begin{itemize}
\item \lstinline!hola!
\item \lstinline!hola12t!
\item \lstinline!_hola!
\item \lstinline!Hola!
\end{itemize}

Python distingue mayúsculas de minúsculas, así que \lstinline!Hola! es
un identificador y \lstinline!hola! es otro identificador.

\begin{observacion}
Por convención, no usaremos identificadores que empiezan con mayúscula.
\end{observacion}

Los siguientes son todos identificadores inválidos de Python:

\begin{itemize}
\item \lstinline!hola a12t!
\item \lstinline!8hola!
\item \lstinline!hola\%!
\item \lstinline!Hola*9!
\end{itemize}

Python reserva 31 palabras para describir la estructura del
programa, y no permite que se usen como identificadores. Cuando en
un programa nos encontramos con que un nombre no es admitido pese
a que su formato es válido, seguramente se trata de una de las
palabras de esta lista, a la que llamaremos de {\it palabras
reservadas}. La lista completa de las palabras reservadas de
Python aparecen en el cuadro \ref{kw}.

\begin{table}[h]
\begin{tabular}[l]{l l l l l}

\lstinline!and! & \lstinline!del! & \lstinline!from! & \lstinline!not! & \lstinline!while! \\
\lstinline!as! & \lstinline!elif! & \lstinline!global! & \lstinline!or! & \lstinline!with! \\
\lstinline!assert! & \lstinline!else! & \lstinline!if! & \lstinline!pass! & \lstinline!yield! \\
 \lstinline!break! & \lstinline!except! & \lstinline!import! & \lstinline!print! \\
\lstinline!class! & \lstinline!exec! & \lstinline!in! & \lstinline!raise! \\
\lstinline!continue! & \lstinline!finally! & \lstinline!is! & \lstinline!return! \\
 \lstinline!def! & \lstinline!for! & \lstinline!lambda! & \lstinline!try! \\
\end{tabular}

\vspace{.5cm} \caption{Las palabras reservadas de Python. Estas
palabras no pueden ser usadas como identificadores.}\label{kw}
\end{table}

\subsection{Expresiones}
Una {\it expresión} es una porción de código Python que produce o
calcula un valor (resultado).

\begin{itemize}
\item Un valor es una expresión (de hecho es la expresión más
sencilla). Por ejemplo el resultado de la expresión \lstinline!111! es
precisamente el número \lstinline!111!.

\item Una variable es una expresión, y el valor que produce es el
que tiene asociado en el estado (si \lstinline!x! $\ra$ 5 en el estado,
entonces el resultado de la expresión \lstinline!x!  es el número 5).

\item Usamos operaciones para combinar expresiones y construir
expresiones más complejas:

\begin{itemize}
\item Si \lstinline!x! es como antes, \lstinline!x + 1! es una expresión cuyo
resultado es 6.

\item Si en el estado \lstinline!millas! $\ra$ 1, \lstinline!pies! $\ra$ 0 y
\lstinline!pulgadas! $\ra$ 0, entonces
\lstinline!1609.344 * millas + 0.3048 * pies + 0.0254 * pulgadas! es una
expresión cuyo resultado es 1609.344.

\item La exponenciación se representa con el símbolo \lstinline!**!. Por
ejemplo, \lstinline!x**3! significa $x^3$.

\item Se pueden usar paréntesis para indicar un orden de
evaluación: \lstinline!((b * b) - (4 * a * c)) / (2 * a)!.

\item Igual que en la matemática, si no hay paréntesis en la
expresión primero se agrupan las exponenciaciones, luego los
productos y cocientes, y luego las sumas y restas.

\item Sin embargo, hay que tener cuidado con lo que sucede con los
cocientes, porque si \lstinline!x! e \lstinline!y! son números enteros,
entonces \lstinline!x / y! se calcula como la {\it división entera}
entre \lstinline!x! e \lstinline!y!:

Si \lstinline!x! se refiere al valor 12 e \lstinline!y! se refiere al valor 9
entonces \lstinline!x / y! se refiere al valor 1.

\item Si \lstinline!x! e \lstinline!y! son números enteros, entonces
\lstinline!x % y!
se calcula como el {\it resto de la división entera} entre \lstinline!x!
e \lstinline!y!:

Si \lstinline!x! se refiere al valor 12 e \lstinline!y! se refiere al valor 9
entonces  \lstinline!x % y!  se refiere al valor 3.

\begin{observacion}
Los números pueden ser tanto enteros (\lstinline!111!, \lstinline!-24!), como
reales (\lstinline!12.5!, \lstinline!12.0!, \lstinline!-12.5!). Dentro de la
computadora se representan de manera diferente, y se comportan de manera
diferente frente a las operaciones.
\end{observacion}

\end{itemize}

\item Conocemos también dos expresiones muy particulares:

\begin{itemize}
\item \lstinline!input!, que devuelve el valor ingresado por teclado tal
como se lo digita (en particular sirve para ingresar valores
numéricos).
\item \lstinline!raw_input!, que devuelve lo ingresado por
teclado como si fuera un texto.
\end{itemize}

\end{itemize}

\ejercicioc{

Aplicando las reglas matemáticas de asociatividad, decidir
cuáles de las siguientes expresiones son iguales entre sí:

\begin{partes}
\item \lstinline!((b * b) - (4 * a * c)) / (2 * a)!,
\item \lstinline!(b * b - 4 * a * c) / (2 * a)!,
\item \lstinline!b * b - 4 * a * c / 2 * a!,
\item \lstinline!(b * b) - (4 * a * c / 2 * a)!
\item \lstinline!1 / 2 * b!
\item \lstinline!b / 2!.
\end{partes}}

\ejercicioc{En Python hagan lo siguiente: Denle a \lstinline!a!, \lstinline!b! y
\lstinline!c!  los valores 10, 100 y 1000 respectivamente y evalúen las
expresiones del ejercicio anterior.}

\ejercicioc{En Python hagan lo siguiente: Denle a \lstinline!a!, \lstinline!b! y
\lstinline!c!  los valores 10.0, 100.0 y 1000.0 respectivamente y evalúen las
expresiones del punto anterior.}

\section{No sólo de números viven los programas} \label{nosolo}

No sólo tendremos expresiones numéricas en un programa Python.
Recuerden el programa que se usó para saludar a muchos amigos:

\begin{codigo-python-sn}
def hola(alguien):
    print "Hola", alguien, "!"
    print "Estoy programando en Python."
\end{codigo-python-sn}

Para invocar a ese programa y hacer que saludara a Ana había que
escribir \lstinline!hola("Ana")!.

La variable \lstinline!alguien! en dicha invocación queda ligada a un
valor que es una cadena de caracteres (letras, dígitos, símbolos,
etc.), en este caso, \lstinline!"Ana"!.

Python usa también una notación con comillas simples para referirse a las
cadenas de caracteres, y habla de \lstinline!'Ana'!.

Como en la sección anterior, veremos las reglas de qué constituyen
expresiones con caracteres:

\begin{itemize}
\item Un valor también acá es una expresión. Por ejemplo el
resultado de la expresión \lstinline!'Ana'! es precisamente
\lstinline!'Ana'!.

\item Una variable es una expresión, y el valor que produce es el
que tiene asociado en el estado (si \lstinline!amiga! $\ra$
\lstinline!'Ana'! en el estado, entonces el resultado de la
expresión \lstinline!amiga! es la cadena \lstinline!'Ana'!).

\item Usamos operaciones para combinar expresiones y construir
expresiones más complejas, pero atención con qué operaciones están
permitidas sobre cadenas:

\begin{itemize}
\item El signo \lstinline!+! no representa la suma sino la concatenación
de cadenas: Si \lstinline!amiga! es como antes, \lstinline!amiga + 'Laura'!
es una expresión cuyo valor es \lstinline!AnaLaura!.

\begin{atencion}
No se pueden sumar cadenas más números.
\begin{codigo-python-sn}
>>> amiga="Ana"
>>> amiga+'Laura'
'AnaLaura'
>>> amiga+3
Traceback (most recent call last):
  File "<stdin>", line 1, in <module>
TypeError: cannot concatenate 'str' and 'int' objects
>>>
\end{codigo-python-sn}
\end{atencion}

\item El signo \lstinline!*! se usa para indicar cuántas veces se repite
una cadena: \lstinline!amiga * 3! es una expresión cuyo valor es
\lstinline!'AnaAnaAna'!.

\begin{atencion}
No se pueden multiplicar cadenas entre sí

\begin{codigo-python-sn}
>>> amiga * 3
'AnaAnaAna'
>>> amiga * amiga
Traceback (most recent call last):
  File "<stdin>", line 1, in <module>
TypeError: can't multiply sequence by non-int of type 'str'
\end{codigo-python-sn}
\end{atencion}

\end{itemize}

\end{itemize}

\section{Instrucciones}

Las {\it instrucciones} son las órdenes que entiende Python. Ya
hemos usado varias instrucciones:

\begin{itemize}

\item hemos mostrado valores por pantalla mediante la instrucción
\lstinline!print!,

\item hemos retornado valores de una función mediante la
instrucción \lstinline!return!,

\item hemos asociado valores con variables y

\item hemos usado un ciclo para repetir un cálculo.
\end{itemize}

\section{Ciclos definidos}
\label{ciclosdef}
Hemos ya usado la instrucción \lstinline!for! en el programa que calcula
cuadrados de enteros en un rango.

\begin{codigo-python-sn}
    for x in range(n1, n2):
        print x*x
\end{codigo-python-sn}

Este ciclo se llama definido porque de entrada, y una vez leídos
\lstinline!n1! y \lstinline!n2!, se sabe exactamente cuántas veces se ejecutará
el cuerpo y qué valores tomará \lstinline!x!.

Un ciclo definido es de la forma

\begin{codigo-python-sn}
for <variable> in <secuencia de valores>:
    <cuerpo>
\end{codigo-python-sn}

En nuestro ejemplo la secuencia de valores es el intervalo de
enteros \lstinline![n1, n1+1, ..., n2-1]! y la variable es \lstinline!x!.

La secuencia de valores se puede indicar como:

\begin{itemize}
\item \lstinline!range(n)!. Establece como secuencia de valores a
 \lstinline![0, 1, ..., n-1]!.

\item \lstinline!range(n1, n2)!. Establece como secuencia de valores a
\lstinline![n1, n1+1, ..., n2-1]!.

\item Se puede definir a mano una secuencia entre corchetes. Por ejemplo,

\begin{codigo-python-sn}
for x in [1, 3, 9, 27]:
        print x*x
\end{codigo-python-sn}

imprimirá los cuadrados de los números 1, 3, 9 y 27.

\end{itemize}

\section{Una guía para el diseño}
En su artículo ``How to program it'', Simon Thompson plantea
algunas preguntas a sus alumnos que son muy útiles para la etapa
de diseño:

\begin{itemize}
\item ¿Han visto este problema antes, aunque sea de manera
ligeramente diferente?

\item ¿Conocen un problema relacionado? ¿Conocen un programa que
puede ser útil?

\item Fíjense en la especificación. Traten de encontrar un
problema que les resulte familiar y que tenga la misma
especificación o una parecida.

\item Acá hay un problema relacionado con el que ustedes tienen y que ya fue
resuelto. ¿Lo pueden usar? ¿Pueden usar sus resultados?  ¿Pueden usar sus
métodos? ¿Pueden agregarle alguna parte auxiliar a ese programa del que ya
disponen?

\item Si no pueden resolver el problema propuesto, traten de
resolver uno relacionado. ¿Pueden imaginarse uno relacionado que
sea más fácil de resolver? ¿Uno más general? ¿Uno más específico?
¿Un problema análogo?

\item ¿Pueden resolver una parte del problema?  ¿Pueden sacar algo
útil de los datos de entrada? ¿Pueden pensar qué información es
útil para calcular las salidas? ¿De qué manera se pueden manipular
las entradas y las salidas de modo tal que estén ``más cerca''
unas de las otras?

\item ¿Usaron todos los datos de entrada? ¿Usaron las condiciones
especiales sobre los datos de entrada que aparecen en el
enunciado? ¿Han tenido en cuenta todos los requisitos que se
enuncian en la especificación?

\end{itemize}

\newpage
\section{Ejercicios}
En los ejercicios a continuación, utilizar los conceptos de análisis,
especificación y diseño antes de realizar la implementación.

\begin{ejercicio} Ciclos definidos
% NOTA: este ejercicio no está incluido en ejercicios.tex.
\begin{partes}
	\item Escribir un ciclo definido para imprimir por pantalla
todos los números entre 10 y 20.
	\item Escribir un ciclo definido que salude por pantalla a
sus cinco mejores amigos/as.
	\item Escribir un programa que use un ciclo definido con
rango numérico, que pregunte los nombres de sus cinco mejores
amigos/as, y los salude.
	\item Escribir un programa que use un ciclo definido con
rango numérico, que pregunte los nombres de sus seis mejores
amigos/as, y los salude.
	\item Escribir un programa que use un ciclo definido con
rango numérico, que averigue a cuántos amigos quieren saludar, les
pregunte los nombres de esos amigos/as, y los salude.
\end{partes}
\end{ejercicio}

\begin{ejercicio}
Escribir un programa que le pregunte al usuario una cantidad de pesos,
una tasa de interés y un número de años y muestre como resultado el monto
final a obtener.  La fórmula a utilizar es:
\begin{displaymath}
C_n = C \times (1+\frac{x}{100})^n
\end{displaymath}
Donde $C$ es el capital inicial, $x$ es la tasa de interés y $n$ es el
número de años a calcular.
\end{ejercicio}

\begin{ejercicio}
Escribir un programa que convierta un valor dado en grados Fahrenheit a
grados Celsius.  Recordar que la fórmula para la conversión es:
$F = \frac{9}{5}C+32$
\end{ejercicio}

\begin{ejercicio}
Utilice el programa anterior para generar una tabla de conversión de
temperaturas, desde 0 °F hasta 120 °F, de 10 en 10.
\end{ejercicio}

\begin{ejercicio}
Escribir un programa que imprima todos los números pares entre dos números
que se le pidan al usuario.
\end{ejercicio}

\begin{ejercicio}
Escribir un programa que reciba un número $n$ por
parámetro e imprima los primeros $n$ números triangulares, junto con su
índice. Los números triangulares se obtienen mediante la suma de los números
naturales desde $1$ hasta $n$.  Es decir, si se piden los primeros 5
números triangulares, el programa debe imprimir:

\begin{verbatim}
1 - 1
2 - 3
3 - 6
4 - 10
5 - 15
\end{verbatim}

{\bf Nota}: hacerlo usando y sin usar la ecuación $\sum_{i=1}^n i = n*(n+1)/2$.
¿Cuál realiza más operaciones?
\end{ejercicio}

\begin{ejercicio}
Escribir un programa que tome una cantidad $m$ de valores ingresados
por el usuario, a cada uno le calcule el factorial e imprima el resultado
junto con el número de orden correspondiente.
\end{ejercicio}

\begin{ejercicio}
Escribir un programa que imprima por pantalla todas las fichas de dominó, de
una por línea y sin repetir.
\end{ejercicio}

\begin{ejercicio}
Modificar el programa anterior para que pueda generar fichas de un juego
que puede tener números de 0 a $n$.
\end{ejercicio}
