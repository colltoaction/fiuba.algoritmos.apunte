% Copyright (C) 2009-2010 Maximiliano Curia <maxy@gnuservers.com.ar>,
%               Margarita Manterola <margamanterola@gmail.com>

% Esta obra est� licenciada de forma dual, bajo las licencias Creative
% Commons:
%  * Atribuci�n-Compartir Obras Derivadas Igual 2.5 Argentina
%    http://creativecommons.org/licenses/by-sa/2.5/ar/
%  * Atribuci�n-Compartir Obras Derivadas Igual 3.0 Unported
%    http://creativecommons.org/licenses/by-sa/3.0/deed.es_AR.
%
% A su criterio, puede utilizar una u otra licencia, o las dos.
% Para ver una copia de las licencias, puede visitar los sitios
% mencionados, o enviar una carta a Creative Commons,
% 171 Second Street, Suite 300, San Francisco, California, 94105, USA.

\chapter{Procesamiento de archivos}

En la unidad \ref{uni:archivos} se explic� como abrir, leer y escribir datos en
los archivos.  En general se quiere poder procesar la informaci�n que contienen
estos archivos, para hacer algo �til con ella.

Dentro de las operaciones a realizar m�s sencillas se encuentran los
denominados {\it filtros}, programas que procesan la entrada l�nea por
l�nea, pudiendo seleccionar qu� l�neas formar�n parte de la salida y
pudiendo aplicar una operaci�n determinada a cada una de estas l�neas antes
de pasarla a la salida.

En esta unidad se indican algunas formas m�s complejas de procesar la
informaci�n le�da.  En particular, dos algoritmos bastante comunes, llamados
\textit {corte de control} y \textit{apareo de archivos}.

\section{Corte de control}

La idea b�sica de este algoritmo es poder analizar informaci�n, generalmente
provista mediante \textit{registros}, agrup�ndolos seg�n diversos criterios.
Como precondici�n se incluye que la informaci�n debe estar ordenada seg�n
los mismos criterios por los que se la quiera agrupar. De modo que si
varios registros tienen el mismo valor en uno de sus \textit{campos}, se
encuentren juntos, formando un grupo.

Se lo utiliza principalmente para realizar reportes que requieren
subtotales, cantidades o promedios parciales u otros valores similares.

El algoritmo consiste en ir recorriendo la informaci�n, de modo que cada vez
que se produzca un cambio en alguno de los campos correspondiente a uno de los
criterios, se ejecutan los pasos correspondientes a la finalizaci�n de un
criterio y el comienzo del siguiente.

\subsection*{Ejemplo}

Supongamos que en un archivo \textbf{csv} tenemos los datos de las ventas de
una empresa a sus clientes y se necesita obtener las ventas por cliente,
mes por mes, con un total por a�o, otro por cliente y uno de las ventas 
totales. El formato est� especificado de la siguiente forma:

\begin{verbatim}
cliente,a�o,mes,d�a,venta
\end{verbatim}

Para poder hacer el reporte como se solicita, el archivo debe estar ordenado en
primer lugar por \verb!cliente!, luego por \verb!a�o!, y luego por \verb!mes!.

Teniendo el archivo ordenado de esta manera, es posible recorrerlo e ir
realizando los subtotales correspondientes, a medida que se los va
obteniendo.

\lstinputlisting[tabsize=4,basicstyle=\small\ttfamily,
title=\texttt{ventas.py} Recorre un archivo de ventas e imprime totales y
subtotales]{src/12_procesamiento/ventas.py}

Se puede ver que para resolver el problema es necesario contar con tres
bucles anidados, que van incrementando la cantidad de condiciones a
verificar.

\begin{observacion}
Las soluciones de corte de control son siempre de estar forma: una serie de
bucles anidados, que incluyen las condiciones del bucle padre y agregan su
propia condici�n, y el movimiento hacia el siguiente registro se realiza en
el bucle con mayor nivel de anidaci�n.
\end{observacion}

\section{Apareo}

As� como el corte de control nos sirve para generar un reporte, el apareo nos
sirve para asociar/relacionar datos que se encuentran en distintos archivos.

La idea b�sica es: a partir de dos archivos (uno principal y otro
relacionado) que tienen alguna informaci�n que los enlace, generar un
tercero (o una salida por pantalla), como una mezcla de los dos.

Para hacer esto es conveniente que ambos archivos est�n ordenados por el valor
que los relaciona. 

\subsection*{Ejemplo}

Por ejemplo, si se tiene un archivo con un listado de alumnos (padr�n,
apellido, nombre, carrera), y otro archivo que contiene las notas de esos
alumnos (padr�n, materia, nota), y se quieren listar todas las notas que
corresponden a cada uno de los alumnos, se lo puede hacer de la siguiente
manera.

\lstinputlisting[tabsize=4,basicstyle=\small\ttfamily,
title=\texttt{notas.py} Recorre un archivo de alumnos y otro de notas e
imprime las notas que corresponden a cada
alumno]{src/12_procesamiento/notas.py}

En el ejemplo anterior usamos apareo de datos para combinar y mostrar
informaci�n, de forma similar se puede utilizar para agregar informaci�n nueva,
borrar informaci�n o modificar datos de la tabla principal. Gran parte de las
bases de datos relacionales basan su funcionamiento en estas funcionalidades.

\section{Resumen}

\begin{itemize}

\item Existen diversas formas de procesar archivos de informaci�n.  Se
puede simplemente filtrar la entrada para obtener una salida, o se pueden
realizar operaciones m�s complejas como el {\bf corte de control} o el {\bf
apareo}

\item El corte de control es una t�cnica de procesamiento de datos
ordenados por diversos criterios, que permite agruparlos para obtener
subtotales.

\item El apareo es una t�cnica de procesamiento que involucra dos archivos
con datos ordenados, y permite generar una salida combinada a partir de
estos dos archivos.

\end{itemize}

