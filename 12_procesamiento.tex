\chapter{Procesamiento de archivos}

En la unidad \ref{uni:archivos} se explicó como abrir, leer y escribir datos en
los archivos.  En general se quiere poder procesar la información que contienen
estos archivos, para hacer algo útil con ella.

Dentro de las operaciones a realizar más sencillas se encuentran los
denominados {\it filtros}, programas que procesan la entrada línea por
línea, pudiendo seleccionar qué líneas formarán parte de la salida y
pudiendo aplicar una operación determinada a cada una de estas líneas antes
de pasarla a la salida.

En esta unidad se muestran algunas formas más complejas de procesar la
información leída, que resultan útiles para procesar grandes cantidades de
datos, evitando cargar la totalidad de los mismos en memoria.

\section{Corte de control}

La idea básica de este algoritmo es poder analizar información, generalmente
provista mediante \textit{registros}, agrupándolos según diversos criterios.
Como precondición se incluye que la información debe estar ordenada según
los mismos criterios por los que se la quiera agrupar. De modo que si
varios registros tienen el mismo valor en uno de sus \textit{campos}, se
encuentren juntos, formando un grupo.

Se lo utiliza principalmente para realizar reportes que requieren
subtotales, cantidades o promedios parciales u otros valores similares.

El algoritmo consiste en ir recorriendo la información, de modo que cada vez
que se produzca un cambio en alguno de los campos correspondiente a uno de los
criterios, se ejecutan los pasos correspondientes a la finalización de un
criterio y el comienzo del siguiente.

\subsection*{Ejemplo}

Supongamos que en un archivo \textbf{csv} tenemos los datos de las ventas de
una empresa a sus clientes y se necesita obtener las ventas por cliente,
mes por mes, con un total por año, otro por cliente y uno de las ventas
totales. El formato está especificado de la siguiente forma:

\begin{codigo-nohl-sn}
cliente,año,mes,día,venta
\end{codigo-nohl-sn}

Para poder hacer el reporte como se solicita, el archivo debe estar ordenado en
primer lugar por \verb!cliente!, luego por \verb!año!, y luego por \verb!mes!.

Teniendo el archivo ordenado de esta manera, es posible recorrerlo e ir
realizando los subtotales correspondientes, a medida que se los va
obteniendo. Se muestra la implementación en el Código \ref{ventas.py}.

\begin{codigo}{ventas.py}{{\bf Ejemplo de corte de control.} Recorre un archivo
	de ventas e imprime totales y subtotales}
\label{ventas.py}
\lstinputlisting{src/12_procesamiento/ventas.py}
\end{codigo}

Se puede ver que para resolver el problema es necesario contar con tres
bucles anidados, que van incrementando la cantidad de condiciones a
verificar.

\begin{observacion}
Las soluciones de corte de control son siempre de esta forma: una serie de
bucles anidados, que incluyen las condiciones del bucle padre y agregan su
propia condición, y el movimiento hacia el siguiente registro se realiza en
el bucle con mayor nivel de anidación.
\end{observacion}

\begin{nota}
En |ventas.py| utilizamos una función que no habíamos mencionado hasta ahora:

\begin{codigo-nohl-sn}
item = next(ventas_csv, None)
\end{codigo-nohl-sn}

La función |next| lee el siguiente registro del archivo CSV y lo devuelve.
Cuando no quedan más registros para leer, devuelve el valor que le pasamos en
el segundo parámetro (|None|).
\end{nota}

\section{Apareo}

Así como el corte de control nos sirve para generar un reporte, el apareo nos
sirve para asociar/relacionar datos que se encuentran en distintos archivos.

La idea básica es: a partir de dos archivos (uno principal y otro
relacionado) que tienen alguna información que los enlace, generar un
tercero (o una salida por pantalla), como una mezcla de los dos.

Para hacer esto es conveniente que ambos archivos estén ordenados por el valor
que los relaciona.

\subsection*{Ejemplo}

Por ejemplo, si se tiene un archivo con un listado de alumnos (padrón,
apellido, nombre, carrera), y otro archivo que contiene las notas de esos
alumnos (padrón, materia, nota), y se quieren listar todas las notas que
corresponden a cada uno de los alumnos, se lo puede hacer como se muestra en el
Código \ref{notas.py}.

\begin{codigo}{notas.py}{{\bf Ejemplo de apareo.} Recorre un archivo de alumnos
	y otro de notas e imprime las notas que corresponden a cada alumno}
\label{notas.py}
\lstinputlisting{src/12_procesamiento/notas.py}
\end{codigo}

En este ejemplo usamos apareo de datos para combinar y mostrar
información. De forma similar se puede utilizar para agregar información nueva,
borrar información o modificar datos de la tabla principal. Gran parte de las
bases de datos relacionales basan su funcionamiento en estas funcionalidades.

\section{Resumen}

\begin{itemize}

\item Existen diversas formas de procesar archivos con grandes cantidades de
	información. Se puede simplemente filtrar la entrada para obtener una
	salida, o se pueden realizar operaciones más complejas como el {\bf corte
	de control} o el {\bf apareo}

\item El corte de control es una técnica de procesamiento de datos
ordenados por diversos criterios, que permite agruparlos para obtener
subtotales.

\item El apareo es una técnica de procesamiento que involucra dos o más
	archivos con datos ordenados, y permite generar una salida combinada a
	partir de los mismos.

\end{itemize}

