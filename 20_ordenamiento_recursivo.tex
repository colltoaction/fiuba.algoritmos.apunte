% Copyright (C) 2008-2010 Rosita Wachenchauzer <rositaw@gmail.com>
%               Margarita Manterola <margamanterola@gmail.com>

% Esta obra está licenciada de forma dual, bajo las licencias Creative
% Commons:
%  * Atribución-Compartir Obras Derivadas Igual 2.5 Argentina
%    http://creativecommons.org/licenses/by-sa/2.5/ar/
%  * Atribución-Compartir Obras Derivadas Igual 3.0 Unported
%    http://creativecommons.org/licenses/by-sa/3.0/deed.es_AR.
%
% A su criterio, puede utilizar una u otra licencia, o las dos.
% Para ver una copia de las licencias, puede visitar los sitios
% mencionados, o enviar una carta a Creative Commons,
% 171 Second Street, Suite 300, San Francisco, California, 94105, USA.

\chapter{Algunos ordenamientos recursivos}

Los métodos de ordenamiento vistos en la unidad anterior eran métodos
iterativos cuyo tiempo estaba relacionado con $N^2$.

En esta unidad veremos dos métodos de ordenamiento, basados
éstos en un planteo recursivo del problema, que nos permitirán obtener el
mismo resultado de forma más eficiente.

\section{Ordenamiento por mezcla, o {\it Merge sort} }

Este método se basa en la siguiente idea:
\begin{enumerate}
\item Si la lista es pequeña (vacía o de tamaño 1) ya está ordenada y
no hay nada que hacer. De lo contrario hacer lo siguiente:
\item Dividir la lista al medio, formando dos sublistas de (aproximadamente) el
mismo tamaño cada una.
\item Ordenar cada una de esas dos sublistas (usando
este mismo método).
\item Una vez que se ordenaron ambas sublistas, intercalarlas de manera ordenada.
\end{enumerate}

Por ejemplo, si la lista original es \lstinline+[6, 7, -1, 0, 5, 2, 3, 8]+
deberemos ordenar recursivamente \lstinline+[6, 7, -1, 0]+ y
\lstinline+[5, 2, 3, 8]+ con lo cual obtendremos \lstinline+[-1, 0, 6, 7]+ y
\lstinline+[2, 3, 5, 8]+.  Si intercalamos ordenadamente las dos listas
ordenadas obtenemos la solución buscada:
\lstinline+[-1, 0, 2, 3, 5, 6, 7, 8]+.

Diseñamos la {\bf función \lstinline!merge_sort(lista)!}:

\begin{enumerate}
\item Si lista es pequeña (vacía o de tamaño 1) ya está ordenada y
no hay nada que hacer. Se devuelve lista tal cual.
\item De lo contrario:
\begin{enumerate}
\item medio = len(lista)/2
\item izq = merge\_sort(lista[:m])
\item der = merge\_sort(lista[m:])
\item Se devuelve merge(izq, der).
\end{enumerate}
\end{enumerate}

Falta sólo diseñar la función \lstinline!merge! que realiza la
intercalación ordenada de dos listas ordenadas (dadas dos listas ordenadas
se debe obtener una nueva lista que resulte de intercalar a ambas de manera
ordenada).

Diseñamos la {\bf función \lstinline!merge(lista1, lista2)!}:

\begin{enumerate}
\item Utilizaremos dos índices, \lstinline!i! y \lstinline!j!, para recorrer
cada una de las dos listas.
\item Utilizaremos una tercera lista, \lstinline!resultado!, donde
almacenaremos el resultado.

\item Mientras \lstinline!i! sea menor que el largo de \lstinline!lista1! y
\lstinline!j! menor que el largo de \lstinline!lista2!, significa que hay
elementos para comparar en ambas listas.

\begin{enumerate}
\item Si el menor es el de \lstinline!lista1!:
\begin{enumerate}
\item Agregar el elemento \lstinline!i! de \lstinline!lista1! al final del
resultado.
\item Avanzar el índice \lstinline!i!.
\end{enumerate}
\item de lo contrario:
\begin{enumerate}
\item Agregar el elemento \lstinline!j! de \lstinline!lista2! al final del
resultado.
\item Avanzar el índice \lstinline!j!.
\end{enumerate}

\end{enumerate}

\item Una vez que una de las dos listas se termina, simplemente hay que
agregar todo lo que queda en la otra al final del resultado.
\end{enumerate}

\begin{codigo}{mergesort.py}{Una implementación de {\it Merge sort}}
\lstinputlisting{src/20_ordenamiento_recursivo/mergesort.py}
\label{src:mergesort}
\end{codigo}

El código resultante del diseño de ambas funciones puede verse en el Código
\ref{src:mergesort}.

\begin{sabias_que}
El método que hemos usado para resolver este problema se llama {\bf División y
Conquista}, y se aplica en las situaciones en las que vale el siguiente
principio:

Para obtener una solución es posible partir el problema en varios subproblemas
de tamaño menor, resolver cada uno de esos subproblemas por separado aplicando
la misma técnica (en nuestro caso ordenar por mezcla cada una de las dos
sublistas), y finalmente juntar estas soluciones parciales en una solución
completa del problema mayor (en nuestro caso la intercalación ordenada de las
dos sublistas ordenadas).

Como siempre sucede con las soluciones recursivas, debemos encontrar un caso
base en el cual no se aplica la llamada recursiva (en nuestro caso la base
sería el paso 1: Si la lista es pequeña (vacía o de tamaño 1) ya está ordenada
y no hay nada que hacer). Además debemos asegurar que siempre se alcanza el
caso base, y en nuestro caso aseguramos eso porque las lista original se divide
siempre en mitades cuando su longitud es mayor que 1.
\end{sabias_que}

\section{¿Cuánto cuesta el {\it Merge sort}?}
Sea $N$ la longitud de la lista. Observamos lo siguiente:
\begin{itemize}

\item Para intercalar dos listas de longitud $N/2$ hace falta recorrer
ambas listas que en total tienen $N$ elementos, por lo que es proporcional
a $N$. Llamemos $a * N$ a ese tiempo.

\item Si llamamos $T(N)$ al tiempo que tarda el algoritmo en ordenar
una lista de longitud $N$, vemos que $T(N) = 2 * T(N/2) + a * N$.

\item Además, cuando la lista es pequeña, la operación es de tiempo
constante: $T(1) = T(0) = b$.
\end{itemize}

Para simplificar la cuenta vamos a suponer que $N = 2^k$.

\begin{eqnarray}
T(N) = T(2^k) &=& 2 * T(2^{k-1}) + a * 2^k \\
              &=& 2 * \left( 2*T(2^{k-2} ) + a * 2^{k-1} \right) + a * 2^k\\
&=& 2^2*T(2^{k-2} ) + a*2^k + a*2^k\\
&&\vdots\\
&=& 2^i* T(2^{k-i})+ i * a * 2^k\\
&&\vdots\\
&=& 2^k*T(1) + k * a * 2^k\\
&=& b * 2^k  + k * a * 2^k
\end{eqnarray}

Pero si $N = 2^k$ entonces $k=\log_2N$, y por lo tanto hemos demostrado
que:

\begin{equation}
T(N) = b N + a N \log_2N.
\end{equation}

Como lo que nos interesa es aproximar el valor, diremos (despreciando el
término de menor orden) que $T(N) \sim N*\log_2N$. Hemos mostrado entonces
un algoritmo que se porta mucho mejor que los que vimos en la unidad
pasada.

Si analizamos el espacio que consume, vemos que a cada paso genera una
nueva lista, de la suma de los tamaños de las dos listas, es decir que
duplica el espacio consumido.

\section{Ordenamiento rápido o {\it Quick sort}}

Veremos ahora el más famoso de los algoritmos recursivos de ordenamiento.
Su fama radica en que en la práctica, con casos reales, es uno de los
algoritmos más eficientes para ordenar.

Este método se basa en la siguiente idea:

\begin{enumerate}
\item Si la lista es pequeña (vacía o de tamaño 1) ya está ordenada y
no hay nada que hacer. De lo contrario hacer lo siguiente:

\item Tomar un elemento de la lista (por ejemplo el primero) al que
llamaremos {\bf pivote} y armar a partir de esa lista tres sublistas: la de
todos los elementos de la lista menores al pivote, la formada sólo por el
pivote, y la de los elementos mayores o iguales al pivote, pero sin
contarlo al pivote.

\item Ordenar cada una de esas tres sublistas (usando este mismo método).

\item Concatenar las tres sublistas ya ordenadas.
\end{enumerate}

Por ejemplo, si la lista original es \lstinline+[6, 7, -1, 0, 5, 2, 3, 8]+
consideramos que el pivote es el primer elemento (el 6) y armamos las
sublistas \lstinline+[-1, 0, 5, 2, 3]+, \lstinline+[6]+ y
\lstinline+[7,8]+. Se ordenan recursivamente \lstinline+[-1, 0, 5, 2, 3]+
(obtenemos \lstinline+[-1, 0, 2, 3, 5]+) y \lstinline+[7, 8]+ (obtenemos la
misma) y concatenamos en el orden adecuado, y así obtenemos
\lstinline+[-1, 0, 2, 3, 5, 6, 7, 8]+.

Para diseñar, vemos que lo más importante es conseguir armar las tres
listas en las que se parte la lista original. Para eso definiremos una
función auxiliar \lstinline!_partition! que recibe una lista no vacía y
devuelve las tres sublistas \lstinline!menores!, \lstinline!medio! y
\lstinline!mayores!  (incluye los iguales, de haberlos) en las que se parte
la lista original usando como pivote al primer elemento.

Contando con la función \lstinline!_partition!, el diseño del {\it Quick sort}
es muy simple:

\begin{enumerate}
\item Si lista es pequeña (vacía o de tamaño 1) ya está ordenada y
no hay nada que hacer. Se devuelve lista tal cual.
\item De lo contrario:
\begin{enumerate}
\item Dividir la lista en tres, usando \lstinline!_partition!.
\item Llamar a \lstinline!quick_sort(menores)!,
\lstinline!quick_sort(mayores)!, y concatenarlo con \lstinline!medio! en el
medio.
\end{enumerate}
\end{enumerate}

Por otro lado, en cuanto a la {\bf función \lstinline!_partition(lista)!}:

\begin{enumerate}
\item Tiene como precondición que la lista es no vacía.
\item Se elige el primer elemento como pivote.
\item Se inicializan como vacías las listas \lstinline!menores! y
\lstinline!mayores!.
\item Para cada elemento de la lista después del primero:
\begin{enumerate}
\item Si es menor que el pivote, se lo agrega a menores.
\item De lo contrario, se lo a agrega a mayores.
\end{enumerate}
\item Devolver menores, [pivote], mayores
\end{enumerate}

Una primera aproximación a este código se puede ver en el Código
\ref{src:quicksort-copia}.

\begin{codigo}{quicksort\_copia.py}{Una primera aproximación al {\it Quick sort}}
\label{src:quicksort-copia}
\lstinputlisting{src/20_ordenamiento_recursivo/quicksort_copia.py}
\end{codigo}

\section{¿Cuánto cuesta el {\it Quick sort}?}

A primera vista, la ecuación del tiempo consumido parece ser la misma que
en el {\it Mergesort}: Una partición que se hace en tiempo lineal más dos
llamadas recursivas a mitades de la lista original.

Pero el problema acá es que la partición tomando como pivote
\lstinline!lista[0]! no siempre parte la lista en mitades: puede suceder (y
ese es el peor caso) que parta a la lista en (\lstinline![]!,
\lstinline![lista[0]]!, \lstinline!lista[1:]!) (esto es lo que pasa cuando
la lista está ordenada de entrada, para el algoritmo presentado), y en ese
caso se comporta como {\it selección}.

En cambio, cuando la lista tiene números ubicados de manera arbitraria
dentro de ella, podemos imaginar un comportamiento parecido al del
Mergesort, y por lo tanto ahí sí $T(N) \sim N * \log_2 N$.

Si analizamos el espacio que consume, el código mostrado en Código
\ref{src:quicksort-copia} crea nuevas listas a cada paso, con lo cual al
igual que el {\it Merge sort} utiliza el doble de memoria.

\section{Una versión mejorada de {\it Quick sort}}

Sin embargo, es posible hacer la partición de otra forma, operando sobre la
misma lista recibida, reubicando los elementos en su interior, de modo que
no se consuma el doble de memoria.

En este caso, tendremos una función \lstinline!_quick_sort!, que será muy
similar al de la vista anteriormente, con la particularidad de que en lugar
de recibir listas cada vez más pequeñas, recibirá los índices de inicio y
fin que indican la porción de la lista sobre la que debe operar.

Habrá, además una función \lstinline!quick_sort!, que recibirá la lista más
parámetros, y se encargará de llamar \lstinline!_quick_sort! con los
índices correspondientes.

Por otro lado, la función \lstinline!_partition! recibirá también los
índices de inicio y fin.  En este caso, la función se encargará de cambiar
de lugar los elementos de la lista, de modo que todos los menores al pivote
se encuentren antes de él y todos los mayores se encuentren después.

Existen varias formas de llevar esto a cabo.  Este es un posible diseño
para la función \lstinline!_partition!:

\begin{enumerate}
\item Elegir el pivote como el primero de los elementos a procesar.
\item Inicializar un índice \lstinline!menores! con el valor del primer
elemento de la porción a procesar.
\item Recorrer los elementos desde el segundo hasta el último a procesar:
\begin{enumerate}
\item Si el elemento es menor al pivote, incrementar el índice
\lstinline!menores! y de ser necesario, intercambiar el elemento para que
pase a ser el último de los menores.
\end{enumerate}
\item Intercambiar el pivote con el último de los menores
\item Devolver la posición del pivote.
\end{enumerate}

El código resultante de este nuevo diseño se reproduce en el Código
\ref{src:quicksort}.

\begin{codigo}{quicksort.py}{Una versión más eficiente de {\it Quicksort}}
\lstinputlisting{src/20_ordenamiento_recursivo/quicksort.py}
\label{src:quicksort}
\end{codigo}

Este código, si bien más complejo, cumple con el objetivo de proveer un
algoritmo de ordenamiento que en el caso promedio tarda
$T(N) \sim N log_2 N$.

\section{Resumen}

\begin{itemize}

\item Los ordenamientos de selección e inserción, presentados en la unidad
anterior son ordenamientos sencillos pero que costosos en cantidad de
intercambios o de comparaciones.  Sin embargo, es posible conseguir
ordenamientos con mejor orden utilizando algoritmos recursivos.

\item El algoritmo {\bf Merge Sort} consiste en dividir la lista a ordenar
hasta que tenga 1 ó 0 elementos y luego combinar la lista de forma ordenada.
De esta manera se logra un tiempo proporcional a $N log N$.  Tiene como
desventaja que siempre utiliza el doble de la memoria requerida por la lista a
ordenar.

\item El algoritmo {\bf Quick Sort} consiste en elegir un elemento, llamado
{\it pivote} y ordenar los elementos de tal forma que todos los menores queden
a la izquierda y todos los mayores a la derecha, y a continuación ordenar de la
misma forma cada una de las dos sublistas formadas.  Puede implementarse de tal
forma que opere sobre la misma lista, sin necesidad de utilizar más memoria.
Tiene como desventaja que si bien en el caso promedio tarda $N log N$, en el
peor caso (según cuál sea el pivote elegido) puede llegar a tardar $N^2$.

\end{itemize}

\newpage
\section{Ejercicios}

\begin{ejercicio}
Escribir una función \verb!merge_sort_3! que funcione igual que el merge sort
pero en lugar de dividir los valores en dos partes iguales, los divida en tres
(asumir que se cuenta con la función \verb!merge(lista_1, lista_2, lista_3)!).
¿Cómo te parece que se va a comportar este método con respecto al merge sort
original?
\end{ejercicio}

\begin{ejercicio}
Mostrar los pasos del ordenamiento de la lista: 6,0,3,2,5,7,4,1 con
los métodos de inserción y con merge sort. ¿Cuáles son las principales
diferencias entre los métodos? ¿Cuál usarías en qué casos?
\end{ejercicio}

\begin{ejercicio}
Ordenar la lista 6,0,3,2,5,7,4,1 usando el método quicksort. Mostrar
el árbol de recursividad explicando las llamadas que se hacen en cada
paso, y el orden en el que se realizan.
\end{ejercicio}
