\chapter*{Prólogo}
\addcontentsline{toc}{chapter}{Prólogo}

Durante mucho tiempo nos preguntamos cómo diseñar un curso de Algoritmos y Programación I
(primera materia de programación de las carreras de Ingeniería en Informática y Licenciatura
en Análisis de Sistemas de la Facultad de Ingeniería de la UBA) que al mismo tiempo fuera
atractivo para los futuros profesionales de la informática, les permitiera aprender a resolver
problemas, y atacara muy directamente el problema de la deserción temprana de estudiantes.


Muchos de los estudiantes que ingresan a primer año de las carreras de informática en nuestro
país lo hacen sin saber programar pese a ser nativos digitales, para los cuales las computadoras
y muchos programas forman parte de su vida cotidiana. El primer curso de programación plantea
entonces varios desafíos: enseñar una metodología para la resolución de problemas, un lenguaje
formal para escribir los programas, y al mismo tiempo hacer que los estudiantes no se sientan
abrumados, tengan éxito en este primer esfuerzo y se sientan atraídos por la posibilidad de
escribir sus propios programas.


En este sentido, la elección del lenguaje de programación para dicho curso no es un tema menor:
el problema radica en elegir un lenguaje que al mismo tiempo sea suficientemente expresivo, tenga
una semántica clara y cuyas complicaciones sintácticas sean mínimas.


Hacía tiempo que buscábamos un lenguaje con todas estas características cuando, durante el debate
posterior a un panel sobre programación en el nivel que tuvo lugar en la conferencia Frontiers in
Engineering Education 2007 organizada por la IEEE, el Dr. John Impagliazzo, de Hofstra University,
relató cómo sus cursos habían pasado de duras experiencias, con altas tasas de deserción, a una
situación muy exitosa, por el solo hecho de haber cambiado el lenguaje de programación de ese curso
de Java a Python. Después de ese estimulante encuentro con Impagliazzo nos pusimos manos a la obra
para diseñar este curso. Fue una grata sorpresa enterarnos también de que el venerable curso de
programación de MIT también había migrado a Python: todo hacía pensar que nuestra elección de lenguaje
no era tan descabellada como se podía pensar.


Este libro pretende entonces ser una muy modesta contribución a la discusión sobre cómo enseñar a
programar en primer año de una carrera de Informática a través de un lenguaje que tenga una suave
curva de aprendizaje, de modo tal que este primer encuentro le resulte a los estudiantes placentero
y exitoso, sin que los detalles del lenguaje los distraiga del verdadero objetivo del curso: la resolución
de problemas mediante computadoras.


Queremos agradecer a la Facultad de Ingeniería (en particular a su Comisión de Publicaciones) y a
Eudeba por la publicación de este libro. Pero también a todos los que apoyaron desde un primer momento
la escritura y mejora continua de lo que fueron durante varios años las notas de nuestro curso de
Algoritmos y Programación I: a la Comisión Curricular de la Licenciatura en Análisis de Sistemas y al
Departamento de Computación (y a su director, Gustavo López) por apoyar la iniciativa de dar este curso
como piloto, a quienes leyeron y discutieron los manuscritos desde sus primeras versiones y colaboraron
con el curso (Melisa Halsband, Alberto Bertogli, Sebastián Santisi, Pablo Antonio, Pablo Najt,
Leandro Ferrigno, Martín Albarracín, Gastón Kleiman, Matías Gavinowich), a quienes fueron primero alumnos y
luego colaboradores de esta experiencia educativa
(Agustín Santiago,
Alejandro Levinas,
Ana Czarnitzki,
Ariel Vergara,
Bruno Merlo Schurmann,
Damián Schenkelman,
Daniela Riesgo,
Daniela Soto,
Débora Martín,
Diego Alfonso,
Emiliano Sorbello,
Ezequiel Genender,
Fabricio Pontes Harsich,
Fabrizio Graffe,
Federico Barrios,
Federico Esteban,
Federico López,
Florencia Álvarez Etcheverry,
Gastón Goncalves,
Gastón Martínez,
Ignacio Garay,
Ignacio Sueiro,
Javier Choque,
Jennifer Woites,
Juan Costamagna,
Lucas Perea,
Manuel Battan,
Manuel Soldini,
Manuel Porto,
Martín Buchwald,
Martín Coll,
Milena Farotto,
Nicolás Poncet,
Pablo Musumeci,
Robinson Fang),
a los amigos que, como Pablo Jacovkis, Elena García y Alejandro Tiraboschi se ofrecieron a
revisar manuscritos y a hacer sugerencias, y a todos los alumnos de nuestro curso de Algoritmos y
Programación I de la FIUBA que, desde 2009, han usado este texto y nos han ayudado a mejorarlo permanentemente.
De todos modos, por supuesto, los errores son nuestra responsabilidad.


\hfill Buenos Aires, diciembre de 2012.

