% Copyright (C) 2009-2010 Maximiliano Curia <maxy@gnuservers.com.ar>,
%               Margarita Manterola <margamanterola@gmail.com>
%               Nicolás Paez <nicopaez@gmail.com>

% Esta obra está licenciada de forma dual, bajo las licencias Creative
% Commons:
%  * Atribución-Compartir Obras Derivadas Igual 2.5 Argentina
%    http://creativecommons.org/licenses/by-sa/2.5/ar/
%  * Atribución-Compartir Obras Derivadas Igual 3.0 Unported
%    http://creativecommons.org/licenses/by-sa/3.0/deed.es_AR.
%
% A su criterio, puede utilizar una u otra licencia, o las dos.
% Para ver una copia de las licencias, puede visitar los sitios
% mencionados, o enviar una carta a Creative Commons,
% 171 Second Street, Suite 300, San Francisco, California, 94105, USA.

\chapter{Manejo de errores y excepciones}

\section{Errores}

En un programa podemos encontrarnos con distintos tipos de errores pero a
grandes rasgos podemos decir que todos los errores pertenecen a una de las
siguientes categorías.

\begin{itemize}

\item Errores de sintaxis: estos errores son seguramente los más simples de
resolver, pues son detectados por el intérprete (o por el compilador, según el
tipo de lenguaje que estemos utilizando) al procesar el código fuente y
generalmente son consecuencia de equivocaciones al escribir el programa. En el
caso de Python estos errores son indicados con un mensaje {\it SyntaxError}.
Por ejemplo, si trabajando con Python intentamos definir una función y en
lugar de {\it def} escribimos {\it dev}.

\item Errores semánticos: se dan cuando un programa, a pesar de no generar
mensajes de error, no produce el resultado esperado. Esto puede deberse, por
ejemplo, a un algoritmo incorrecto o a la omisión de una sentencia.

\item Errores de ejecución: estos errores aparecen durante la ejecución del
programa y su origen puede ser diverso. En ocasiones pueden producirse por un
uso incorrecto del programa por parte del usuario, por ejemplo si el usuario
ingresa una cadena cuando se espera un número. En otras ocasiones pueden
deberse a errores de programación, por ejemplo si una función intenta acceder
a la quinta posición de una lista de 3 elementos o realizar una división por
cero. Una causa común de errores de ejecución que generalmente excede al
programador y al usuario, son los recursos externos al programa, por ejemplo
si el programa intenta leer un archivo y el mismo se encuentra dañado.

\end{itemize}

Tanto a los errores de sintaxis como a los semánticos se los puede detectar y
corregir durante la construcción del programa ayudados por el intérprete y
la ejecución de pruebas. Pero no ocurre esto con los errores de ejecución ya
que no siempre es posible saber cuando ocurrirán y puede resultar muy complejo
(o incluso casi imposible) reproducirlos. Es por ello que el resto de la
unidad nos centraremos en cómo preparar nuestros programas para lidiar con
este tipo de errores.

\section{Excepciones}

Los errores de ejecución son llamados comúnmente {\it excepciones} y por eso
de ahora en más utilizaremos ese nombre. Durante la ejecución de un programa,
si dentro de una función surge una excepción y la función no la maneja, la
excepción se propaga hacia la función que la invocó, si esta otra tampoco la
maneja, la excepción continua propagándose hasta llegar a la función inicial
del programa y si esta tampoco la maneja se interrumpe la ejecución del
programa. Veamos entonces como manejar excepciones.

\subsection{Manejo de excepciones}

Para el manejo de excepciones los lenguajes proveen ciertas palabras
reservadas, que nos permiten manejar las excepciones que puedan surgir y
tomar acciones de recuperación para evitar la interrupción del programa o,
al menos, para realizar algunas acciones adicionales antes de interrumpir
el programa.

En el caso de Python, el manejo de excepciones se hace mediante los
bloques que utilizan las sentencias \lstinline!try!, \lstinline!except! y
\lstinline!finally!.

Dentro del bloque \lstinline!try! se ubica todo el código que pueda llegar
a {\it levantar} una excepción, se utiliza el término {\it levantar} para
referirse a la acción de generar una excepción.

A continuación se ubica el bloque \lstinline!except!, que se encarga de
capturar la excepción y nos da la oportunidad de procesarla mostrando por
ejemplo un mensaje adecuado al usuario.

Veamos qué sucede si se quiere realizar una división por cero:

\begin{codigo-python-sn}
>>> dividendo = 5
>>> divisor = 0
>>> dividendo / divisor
Traceback (most recent call last):
  File "<stdin>", line 1, in <module>
ZeroDivisionError: integer division or modulo by zero
\end{codigo-python-sn}

En este caso, se levantó la excepción \lstinline!ZeroDivisionError! cuando se
quiso hacer la división.  Para evitar que se levante la excepción y se detenga
la ejecución del programa, se utiliza el bloque
\lstinline!try!-\lstinline!except!.

\begin{codigo-python-sn}
>>> try:
...     cociente = dividendo / divisor
... except:
...     print "No se permite la división por cero"
...
No se permite la división por cero
\end{codigo-python-sn}

Dado que dentro de un mismo bloque \lstinline!try! pueden producirse
excepciones de distinto tipo, es posible utilizar varios bloques
\lstinline!except!, cada uno para capturar un tipo distinto de excepción.

Esto se hace especificando a continuación de la sentencia
\lstinline!except! el nombre de la excepción que se pretende capturar. Un
mismo bloque \lstinline!except! puede atrapar varios tipos de excepciones,
lo cual se hace especificando los nombres de la excepciones separados por
comas a continuación de la palabra \lstinline!except!. Es importante
destacar que si bien luego de un bloque \lstinline!try! puede haber varios
bloques \lstinline!except!, se ejecutará, a lo sumo, uno de ellos.

\begin{codigo-python-sn}
try:
	# aquí ponemos el código que puede lanzar excepciones
except IOError:
	# entrará aquí en caso que se haya producido
	# una excepción IOError
except ZeroDivisionError:
	# entrará aquí en caso que se haya producido
	# una excepción ZeroDivisionError
except:
	# entrará aquí en caso que se haya producido
	# una excepción que no corresponda a ninguno
	# de los tipos especificados en los except previos
\end{codigo-python-sn}

Como se muestra en el ejemplo precedente también es posible utilizar una
sentencia \lstinline!except! sin especificar el tipo de excepción a
capturar, en cuyo caso se captura cualquier excepción, sin importar su
tipo. Cabe destacar, también, que en caso de utilizar una sentencia
\lstinline!except! sin especificar el tipo, la misma debe ser siempre la
última de las sentencias \lstinline!except!, es decir que el siguiente
fragmento de código es incorrecto.

\begin{codigo-python-sn}[numbers=none]
try:
	# aquí ponemos el código que puede lanzar excepciones
except:
	# ERROR de sintaxis, esta sentencia no puede estar aquí,
	# sino que debería estar luego del except IOError.
except IOError:
	# Manejo de la excepción de entrada/salida
\end{codigo-python-sn}

Finalmente, puede ubicarse un bloque \lstinline!finally! donde se escriben
las sentencias de finalización, que son típicamente acciones de limpieza.
La particularidad del bloque \lstinline!finally! es que se ejecuta siempre,
haya surgido una excepción o no. Si hay un bloque \lstinline!except!, no es
necesario que esté presente el \lstinline!finally!, y es posible tener un
bloque \lstinline!try! sólo con \lstinline!finally!, sin
\lstinline!except!. \\

Veamos ahora como es que actúa Python al encontrarse con estos bloques. Python
comienza a ejecutar las instrucciones que se encuentran dentro de un bloque
\lstinline!try! normalmente. Si durante la ejecución de esas instrucciones
se levanta una excepción, Python interrumpe la ejecución en el
punto exacto en que surgió la excepción y pasa a la ejecución del bloque
\lstinline!except! correspondiente.

Para ello, Python verifica uno a uno los bloques \lstinline!except! y si
encuentra alguno cuyo tipo haga referencia al tipo de excepción levantada,
comienza a ejecutarlo. Sino encuentra ningún bloque del tipo
correspondiente pero hay un bloque \lstinline!except! sin tipo, lo
ejecuta. Al terminar de ejecutar el bloque correspondiente, se pasa a la
ejecución del bloque \lstinline!finally!, si se encuentra definido.

Si, por otra parte, no hay problemas durante la ejecución del bloque
\lstinline!try!, se completa la ejecución del bloque, y luego se pasa
directamente a la ejecución del bloque \lstinline!finally! (si es que está
definido).

Bajemos todo esto a un ejemplo concreto, supongamos que nuestro programa
tiene que procesar cierta información ingresada por el usuario y guardarla
en un archivo. Dado que el acceso a archivos puede levantar
excepciones, siempre deberíamos colocar el código de manipulación de
archivos dentro de un bloque \lstinline!try!. Luego deberíamos
colocar un bloque \lstinline!except! que atrape una excepción del tipo
\lstinline!IOError!, que es el tipo de excepciones que lanzan la funciones
de manipulación de archivos. Adicionalmente podríamos agregar un bloque
\lstinline!except! sin tipo por si surge alguna otra excepción.  Finalmente
deberíamos agregar un bloque \lstinline!finally! para cerrar el archivo,
haya surgido o no una excepción.

\begin{codigo-python-sn}
try:
	archivo = open("miarchivo.txt")
	# procesar el archivo
except IOError:
	print "Error de entrada/salida."
	# realizar procesamiento adicional
except:
	# procesar la excepción
finally:
	# si el archivo no está cerrado hay que cerrarlo
	if not(archivo.closed):
		archivo.close()
\end{codigo-python-sn}

\subsection{Procesamiento y propagación de excepciones}

Hemos visto cómo atrapar excepciones, es necesario ahora que veamos qué se
supone que hagamos al atrapar una excepción. En primer lugar podríamos
ejecutar alguna lógica particular del caso como: cerrar un archivo,
realizar una procesamiento alternativo al del bloque \lstinline!try!, etc.
Pero más allá de esto tenemos algunas opciones genéricas que consisten en:
dejar constancia de la ocurrencia de la excepción, propagar la excepción o,
incluso, hacer ambas cosas.

Para dejar constancia de la ocurrencia de la excepción, se puede escribir
en un archivo de log o simplemente mostrar un mensaje en pantalla.
Generalmente cuando se deja constancia de la ocurrencia de una excepción se
suele brindar alguna información del contexto en que ocurrió la excepción,
por ejemplo: tipo de excepción ocurrida, momento en que ocurrió la
excepción y cuáles fueron las llamadas previas a la excepción. El objetivo
de esta información es facilitar el diagnóstico en caso de que alguien deba
corregir el programa para evitar que la excepción siga apareciendo.

Es posible, por otra parte, que luego de realizar algún procesamiento
particular del caso se quiera que la excepción se propague hacia la función
que había invocado a la función actual. Para hacer esto Python nos brinda
la instrucción \lstinline!raise!.

Si se invoca esta instrucción dentro de un bloque \lstinline!except!, sin
pasarle parámetros, Python levantará la excepción atrapada por ese bloque.

También podría ocurrir que en lugar de propagar la excepción tal cual fue
atrapada, quisiéramos lanzar una excepción distinta, más significativa para
quien invocó a la función actual y que posiblemente contenga cierta
información de contexto. Para levantar una excepción de cualquier tipo,
utilizamos también la sentencia \lstinline!raise!, pero indicándole el tipo
de excepción que deseamos lanzar y pasando a la excepción los parámetros
con información adicional que queramos brindar.

El siguiente fragmento de código muestra este uso de \lstinline!raise!.

\begin{codigo-python-sn}
def dividir(dividendo, divisor):
	try:
		resultado = dividendo / divisor
		return resultado
	except ZeroDivisionError:
		raise ZeroDivisionError("El divisor no puede ser cero")
\end{codigo-python-sn}

\subsection{Acceso a información de contexto}

Para acceder a la información de contexto estando dentro de un bloque
\lstinline!except!  existen dos alternativas. Se puede utilizar la función
\lstinline!exc_info!  del módulo \lstinline!sys!. Esta función devuelve una
tupla con información sobre la última excepción atrapada en un bloque
\lstinline!except!. Dicha tupla contiene tres elementos: el tipo de
excepción, el valor de la excepción y las llamadas realizadas.

Otra forma de obtener información sobre la excepción es utilizando la
misma sentencia \lstinline!except!, pasándole un identificador para que
almacene una referencia a la excepción atrapada.

\begin{codigo-python-sn}
try:
	# código que puede lanzar una excepción
except Exception, ex:
	# procesamiento de la excepción cuya información
	# es accesible a través del identificador ex
\end{codigo-python-sn}

% TODO:
% abuso de excepciones
% La idea sería hablar de uso de excepciones para manejar casos excepcionales
% y no para todo.

\begin{sabias_que}
En otros lenguajes, como el lenguaje Java, si una función puede lanzar una
excepción en alguna situación, la o las excepciones que lance deben formar
parte de la declaración de la función y quien invoque dicha función está
obligado a hacerlo dentro de un bloque \lstinline!try! que la atrape.

En Python, al no tener esta obligación por parte del lenguaje debemos tener
cuidado de atrapar las excepciones probables, ya que de no ser así los
programas se terminarán inesperadamente.
\end{sabias_que}

\section{Validaciones}

Las validaciones son técnicas que permiten asegurar que los valores con los
que se vaya a operar estén dentro de determinado dominio.

Estas técnicas son particularmente importantes al momento de utilizar entradas
del usuario o de un archivo (o entradas externas en general) en nuestro
código, y también se las utiliza para comprobar precondiciones. Al
uso intensivo de estas técnicas se lo suele llamar {\it programación
defensiva}.

Si bien quien invoca una función debe preocuparse de cumplir con las
precondiciones de ésta, si las validaciones están hechas correctamente pueden
devolver información valiosa para que el invocante pueda actuar en
consecuencia.

Hay distintas formas de comprobar el dominio de un dato. Se puede comprobar
el contenido; que una variable sea de un tipo en particular; o que el dato
tenga determinada característica, como que deba ser ``comparable'', o
``iterable''.

También se debe tener en cuenta qué hará nuestro código cuando una
validación falle, ya que queremos darle información al invocante que le sirva
para procesar el error. El error producido tiene que ser fácilmente
reconocible.  En algunos casos, como por ejemplo cuando se quiere devolver
una posición, devolver \lstinline!-1! nos puede asegurar que el invocante
lo vaya a reconocer. En otros casos, levantar una excepción es una solución
más elegante.

En cualquier caso, lo importante es que el resultado generado por nuestro
código cuando funciona correctamente y el resultado generado cuando falla
debe ser claramente distinto. Por ejemplo, si el código debe devolver un
elemento de una secuencia, no es una buena idea que devuelva
\lstinline!None! en el caso de que la secuencia esté vacía, ya que
\lstinline!None! es un elemento válido dentro de una secuencia.

\subsection{Comprobaciones por contenido}

Cuando queremos validar que los datos provistos a una porción de código
contengan la información apropiada, ya sea porque esa información la ingresó
un usuario, fue leída de un archivo, o porque por cualquier motivo es posible
que sea incorrecta, es deseable comprobar que el contenido de las variables a
utilizar estén dentro de los valores con los que se puede operar.

Estas comprobaciones no siempre son posibles, ya que en ciertas situaciones
puede ser muy costoso corroborar las precondiciones de una función. Es por
ello que este tipo de comprobaciones se realizan sólo cuando sea posible.

Por ejemplo, la función factorial está definida para los números naturales
incluyendo el 0. Es posible utilizar \lstinline!assert! (que es otra forma de
levantar una excepción) para comprobar las precondiciones de factorial.

\begin{codigo-python}
def factorial(n):
	""" Calcula el factorial de n.
	Pre: n debe ser un entero, mayor igual a 0
	Post: se devuelve el valor del factorial pedido
	"""
	assert n >= 0, "n debe ser mayor igual a 0"
	fact=1
	for i in xrange(2,n+1):
		fact*=i
	return fact
\end{codigo-python}

\subsection{Entrada del usuario}

En el caso particular de una porción de código que trate con entrada del
usuario, no se debe asumir que el usuario vaya a ingresar los datos
correctamente, ya que los seres humanos tienden a cometer errores al ingresar
información.

Por ejemplo, si se desea que un usuario ingrese un número, no se debe asumir
que vaya a ingresarlo correctamente. Se lo debe guardar en una cadena y luego
convertir a un número, es por eso que es recomendable el uso de la
función \lstinline!raw_input! ya que devuelve una cadena que puede ser
procesada posteriormente.

\begin{codigo-python-sn}
def lee_entero():
    """ Solicita un valor entero y lo devuelve.
        Si el valor ingresado no es entero, lanza una excepción. """
    valor = raw_input("Ingrese un número entero: ")
    return int(valor)
\end{codigo-python-sn}

Esta función devuelve un valor entero, o lanza una excepción si la conversión
no fue posible.  Sin embargo, esto no es suficiente.  En el caso en el que el
usuario no haya ingresado la información correctamente, es necesario volver a
solicitarla.

\begin{codigo-python-sn}
def lee_entero():
    """ Solicita un valor entero y lo devuelve.
        Mientras el valor ingresado no sea entero, vuelve a solicitarlo. """
    while True:
        valor = raw_input("Ingrese un número entero: ")
		try:
			valor = int(valor)
            return valor
        except ValueError:
            print "ATENCIÓN: Debe ingresar un número entero."
\end{codigo-python-sn}

Podría ser deseable, además, poner un límite a la cantidad máxima de intentos
que el usuario tiene para ingresar la información correctamente y, superada
esa cantidad máxima de intentos, levantar una excepción para que sea manejada
por el código invocante.

\begin{codigo-python-sn}
def lee_entero():
    """ Solicita un valor entero y lo devuelve.
        Si el valor ingresado no es entero, da 5 intentos para ingresarlo
        correctamente, y de no ser así, lanza una excepción. """
    intentos = 0
    while intentos < 5:
        valor = raw_input("Ingrese un número entero: ")
        try:
            valor = int(valor)
            return valor
        except ValueError:
            intentos += 1
    raise ValueError, "Valor incorrecto ingresado en 5 intentos"
\end{codigo-python-sn}

Por otro lado, cuando la entrada ingresada sea una cadena, no es esperable que
el usuario la vaya a ingresar en mayúsculas o minúsculas, ambos casos deben
ser considerados.

\begin{codigo-python-sn}
def lee_opcion():
    """ Solicita una opción de menú y la devuelve. """
    while True:
        print "Ingrese A (Altas) - B (Bajas) - M (Modificaciones): ",
        opcion = raw_input().upper()
        if opcion in ["A", "B", "M"]:
            return opcion
\end{codigo-python-sn}

\subsection{Comprobaciones por tipo}

En esta clase de comprobaciones nos interesa el tipo del dato que vamos a
tratar de validar, Python nos indica el tipo de una variable usando la
función \lstinline!type(variable)!. Por ejemplo, para comprobar que una
variable contenga un tipo entero podemos hacer:

\begin{codigo-python-sn}
if type(i) != int:
	raise TypeError, "i debe ser del tipo int"
\end{codigo-python-sn}

Sin embargo, ya hemos visto que tanto las listas como las tuplas y las
cadenas son secuencias, y muchas de las funciones utilizadas puede utilizar
cualquiera de estas secuencias. De la misma manera, una función puede
utilizar un valor numérico, y que opere correctamente ya sea entero,
flotante, o complejo.

Es posible comprobar el tipo de nuestra variable contra una secuencia de
tipos posibles.

\begin{codigo-python-sn}
if type(i) not in (int, float, long, complex):
	raise TypeError, "i debe ser numérico"
\end{codigo-python-sn}

Si bien esto es bastante más flexible que el ejemplo anterior, también puede
ser restrictivo ya que -como se verá más adelante- cada programador puede
definir sus propios tipos utilizando como base los que ya están definidos.
Con este código se están descartando todos los tipos que se basen en
\lstinline!int!, \lstinline!float!,
\lstinline!long! o \lstinline!complex!.

Para poder incluir estos tipos en la comprobación a realizar, Python nos
provee de la función \lstinline!isinstance(variable, tipos)!.

\begin{codigo-python-sn}
if not isinstance(i, (int, float, long, complex) ):
	raise TypeError, "i debe ser numérico"
\end{codigo-python-sn}

Con esto comprobamos si una variable es de determinado tipo o subtipo de
éste. Esta opción es bastante flexible, pero existen aún más opciones.

\begin{atencion}
Hacer comprobaciones sobre los tipos de las variables suele resultar
demasiado restrictivo, ya que es muy posible que una porción de código que
opere con un tipo en particular funcione correctamente con otros tipos de
variables que se comporten de forma similar.

Es por eso que hay que tener mucho cuidado al limitar el uso de una
variable por su tipo, y en muchos casos es preferible limitarlas por sus
propiedades, como el ejemplo anterior, en que se requería que se pudiera
convertir a un entero.
\end{atencion}

Para la mayoría de los tipos básicos de Python existe una función que se
llama de la misma manera que el tipo que devuelve un elemento de ese tipo,
por ejemplo, \lstinline!int()! devuelve 0, \lstinline!dict()! devuelve
\lstinline!{}! y así. Además, estas funciones suelen poder recibir un
elemento de otro tipo para tratar de convertirlo, por ejemplo,
\lstinline!int(3.0)! devuelve 3, \lstinline!list("Hola")! devuelve
\lstinline!['H', 'o', 'l', 'a']!.

Usando está conversión conseguimos dos cosas: podemos convertir un tipo
recibido al que realmente necesitamos, a la vez que tenemos una copia de
este, dejando el original intacto, que es importante cuando estamos
tratando con tipos mutables.

Por ejemplo, si se quiere contar con una función de división entera que
pueda recibir diversos parámetros, podría hacerse de la siguiente manera.

\begin{codigo-python-sn}
def division_entera(x,y):
    """ Calcula la división entera después de convertir los parámetros a
    enteros. """
    try:
        dividendo = int(x)
        divisor = int(y)
        return dividendo/divisor
    except ValueError:
        raise ValueError, "x e y deben poder convertirse a enteros"
    except ZeroDivisionError:
        raise ZeroDivisionError, "y no puede ser cero"
\end{codigo-python-sn}

De esta manera, la función \lstinline!division_entera! puede ser llamada
incluso con cadenas que contengan expresiones enteras. Que este
comportamiento sea deseable o no, depende siempre de cada caso.

\subsection{Comprobaciones por características}

Otra posible comprobación, dejando de lado los tipos, consiste en verificar
si una variable tiene determinada característica o no. Python promueve este
tipo de programación, ya que el mismo intérprete utiliza este tipo de
comprobaciones. Por ejemplo, para imprimir una variable, Python convierte
esa variable a una cadena, no hay en el intérprete una verificación para
cada tipo, sino que busca una función especial, llamada
\lstinline!__str__!, en la variable a imprimir, y si existe, la utiliza
para convertir la variable a una cadena.

\begin{sabias_que}
Python utiliza la idea de {\it duck typing}, que viene del concepto de que
si algo parece un pato, camina como un pato y grazna como un pato,
entonces, se lo puede considerar un pato.

Esto se refiere a no diferenciar las variables por los tipos a los que
pertenecen, sino por las funciones que tienen.
\end{sabias_que}

Para comprobar si una variable tiene o no una función Python provee la
función \lstinline!hasattr(variable, atributo)!, donde \lstinline!atributo!
puede ser el nombre de la función o de la variable que se quiera verificar.
Se verá más sobre atributos en la unidad de Programación Orientada a
Objetos.

Por ejemplo, existe la función \lstinline!__add__! para realizar
operaciones de suma entre elementos.  Si se quiere corroborar si un
elemento es sumable, se lo haría de la siguiente forma.

\begin{codigo-python-sn}
if not hasattr(i,"__add__"):
	raise TypeError, "El elemento no es sumable"
\end{codigo-python-sn}

Sin embargo, que el atributo exista no quiere decir que vaya a funcionar
correctamente en todos los casos. Por ejemplo, tanto las cadenas como los
números definen su propia ``suma'', pero no es posible sumar cadenas y
números, de modo que en este caso sería necesario tener en cuenta una
posible excepción.

Por otro lado, en la mayoría de los casos se puede aplicar la frase: {\it
es más fácil pedir perdón que permiso}, atribuída a la programadora Grace
Hopper. Es decir, en este caso es más sencillo hacer la suma dentro de un
bloque \lstinline!try! y manejar la excepción en caso de error, que saber
cuáles son los detalles de la implementación de \lstinline!__add__! de cada
tipo interactuante.

\section{Resumen}

\begin{itemize}
\item Los errores que se pueden presentar en un programa son: de sintaxis
(detectados por el intérprete), de semántica (el programa no funciona
correctamente), o de ejecución ({\it excepciones}).
\item Cuando el código a ejecutar pueda producir una excepción es deseable
encerrarlo en los bloques correspondientes para actuar en consecuencia.
\item Si una función no contempla la excepción, ésta es levantada a la
función invocante, si ésta no la contempla, la excepción se pasa a la
invocante, hasta que se llega a una porción de código que contemple la
excepción, o bien se interrumpe la ejecución del programa.
\item Cuando una porción de código puede levantar diversos tipos de
excepciones, es deseable tratarlas por separado, si bien es posible
tratarlas todas juntas.
\item Cuando se genera una excepción es importante actuar en consecuencia,
ya sea mostrando un mensaje de error, guardándolo en un archivo, o
modificando el resultado final de la función.
\item Antes de actuar sobre un dato en una porción de código, es deseable
corroborar que se lo pueda utilizar, se puede validar su contenido, su tipo
o sus atributos.
\item Cuando no es posible utilizar un dato dentro de una porción de
código, es importante informar el problema al código invocante, ya sea
mediante una excepción o mediante un valor de retorno especial.
\end{itemize}

\begin{referencia_python}

\begin{sintaxis}{\lstinline{try: ... except:}}
\begin{codigo-python-sn}
try:
	# código
except [tipo_de_excepción [, variable]]:
	# manejo de excepción
\end{codigo-python-sn}

Puede tener tantos \lstinline!except! como sea necesario, el último puede
no tener un tipo de excepción asociado.

Si el código dentro del bloque \lstinline!try! levanta una excepción, se
ejecuta el código dentro del bloque \lstinline!except! correspondiente.
\end{sintaxis}

\begin{sintaxis}{\lstinline{try: ... finally:}}
\begin{lstlisting}[numbers=none]
try:
	# código
finally:
	# código de limpieza
\end{lstlisting}

El código que se encuentra en el bloque \lstinline!finally! se ejecuta al
finalizar el código que se encuentra en el bloque \lstinline!try!, sin
importar si se levantó o no una excepción.
\end{sintaxis}

\begin{sintaxis}{\lstinline{try: ... except: ... finally:}}
\begin{codigo-python-sn}
try:
	# código
except [tipo_de_excepción [, variable]]:
	# manejo de excepción
finally:
	# código de limpieza
\end{codigo-python-sn}

Es una combinación de los otros dos casos.  Si el código del bloque
\lstinline!try! levanta una excepción, se ejecutará el manejador
correspondiente y, sin importar lo que haya sucedido, se ejecutará el
bloque \lstinline!finally! al concluir los otros bloques.
\end{sintaxis}


\begin{sintaxis}{\lstinline{raise [excepción[, mensaje]]}}
Levanta una excepción, para interrumpir el código de la función invocante.

Puede usarse sin parámetros, para levantar la última excepción atrapada.
El primer parámetro corresponde al tipo de excepción a levantar.
El mensaje es opcional, se utiliza para dar más información sobre el error
acontecido.
\end{sintaxis}

\end{referencia_python}

\section{Apéndice}
A continuación se muestran dos ejemplos de implementación de un programa que
calcula la serie de Fibonacci para números menores a 20.

En primer lugar se muestra una implementación sin uso de excepciones, con
las herramientas vistas antes de esta unidad.

\begin{codigo-python}
def calcularFibonacciSinExcepciones(n):
	if (n>=20) or (n<=0):
		print ''' Ha ingresado un valor incorrecto.
El valor debe ser número entero mayor a cero y menor a 20'''
		return
	salida=[]
	a,b = 0,1
	for x in range(n):
		salida.append(b)
		a, b = b, a+b
	return salida

def mainSinExcepciones():
	input = raw_input('Ingrese n para calcular Fibonacci(n):')
	n = int(input)
	print calcularFibonacciSinExcepciones(n)
\end{codigo-python}

A continuación un código que utiliza excepciones para manejar la entrada de
mejor manera.

\begin{codigo-python}
def calcularFibonacciConExcepciones(n):
	try:
		assert(n>0)
		assert(n<20)
	except AssertionError:
		raise ValueError
	a=0
	b=1
	salida = []
	for x in range(n):
		salida.append(b)
		a, b = b, a+b
	return salida

def mainConExcepciones():
	try:
		input = raw_input('Ingrese n para calcular Fibonacci(n):')
		n = int(input)
		print calcularFibonacci2(n)
	except ValueError:
		print '''Ha ingresado un valor incorrecto.
El valor debe ser un número entero mayor a cero y menor a 20'''
\end{codigo-python}
